% Setup and import
\documentclass[12pt,a4paper]{report}
\usepackage[english, italian]{babel}
\usepackage{newlfont}
\usepackage{algorithm}
\usepackage{algpseudocode}
\usepackage{amsmath}
\usepackage{listings}
\usepackage{tikz}
\textwidth=450pt\oddsidemargin=0pt

% Dedication template
\newenvironment{dedication}
{
    \clearpage            % we want a new page
    \thispagestyle{empty} % no header and footer
    \vspace*{\stretch{1}} % some space at the top
    \itshape              % the text is in italics
    \raggedleft           % flush to the right margin
}
{
    \par                  % end the paragraph
    \vspace{\stretch{3}}  % space at bottom is three times that at the top
    \clearpage            % finish off the page
}


% Definition init
\newtheorem{definition}{Definition}[chapter]
\newtheorem{remark}{Remark}[definition]

% Setup algorithm numeration to follow the definition and remark ones
\renewcommand{\thealgorithm}{\arabic{chapter}.\arabic{algorithm}} 

% Setup listings code style
\definecolor{codegreen}{rgb}{0,0.6,0}
\definecolor{codegray}{rgb}{0.5,0.5,0.5}
\definecolor{codepurple}{rgb}{0.58,0,0.82}
\definecolor{backcolour}{rgb}{0.95,0.95,0.92}
\lstdefinestyle{mystyle}{
    backgroundcolor=\color{backcolour},
    commentstyle=\color{codegreen},
    keywordstyle=\color{magenta},
    numberstyle=\tiny\color{codegray},
    stringstyle=\color{codepurple},
    basicstyle=\ttfamily\footnotesize,
    breakatwhitespace=false,
    breaklines=false,
    captionpos=b,
    keepspaces=true,
    numbers=left,
    numbersep=5pt,
    showspaces=false,
    showstringspaces=false,
    showtabs=false,
    tabsize=2
}
\lstset{style=mystyle}


% ForEach construct definition
\algnewcommand\algorithmicforeach{\textbf{for each}}
\algdef{S}[FOR]{ForEach}[1]{\algorithmicforeach\ #1\ \algorithmicdo}

% Tikz submodule import
\usetikzlibrary{automata, positioning, arrows}

% Document start 
\begin{document}


% Title page
\begin{titlepage}
    \begin{center}
        {{\Large{\textsc{Alma Mater Studiorum $\cdot$ Universit\`a di
                            Bologna}}}} \rule[0.1cm]{15.8cm}{0.1mm}
        \rule[0.5cm]{15.8cm}{0.6mm}
        {\small{\bf SCUOLA DI SCIENZE\\
                Corso di Laurea in Informatica }}
    \end{center}
    \vspace{15mm}
    \begin{center}
        {\LARGE{\bf Choreia: A Static Analyzer }}\\
        \vspace{3mm}
        {\LARGE{\bf to generate Choreography Automata}}\\
        \vspace{3mm}
        {\LARGE{\bf from Go source code}}\\
    \end{center}
    \vspace{40mm}
    \par
    \noindent
    \begin{minipage}[t]{0.47\textwidth}
        {\large{\bf Relatore:\\
                Chiar.mo Prof.\\
                Ivan Lanese}}
    \end{minipage}
    \hfill
    \begin{minipage}[t]{0.47\textwidth}\raggedleft
        {\large{\bf Presentata da:\\
                Enea Guidi}}
    \end{minipage}
    \vspace{20mm}
    \begin{center}
        {\large{\bf Sessione III\\
                Anno Accademico 2020/2021 }}
    \end{center}
\end{titlepage}


% Dedication page
\begin{dedication}
    Alla mia famiglia e agli amici che\\
    mi hanno accompagnato in questo viaggio
\end{dedication}


% Abstract
\begin{abstract}
    Le Choreographies sono un paradigma emergente per la descrizione dei sistemi concorrenti che sta prendendo piede negli utlimi anni. Lo scopo principale è quello di fornire al programmatore uno strumento che permetta di capire in maniera immediata la \emph{coreografia} dei participanti all'interno del sistema e come questi interagiscano tra loro. Partendo dai singoli partecipanti, e le loro Choreographies \emph{locali}, è possibile ricomporre in maniera bottom-up l'intera Choreography \emph{globale} del sistema.
    Un ulteriore vantaggio delle Choreographies è che, quando rispettano alcune proprietà definite, permettono di fare assunzioni sull'assenza di tipici problemi di concorrenza quali Deadlocks, Liveness e Race Conditions.
    Esistono vari modelli formali di Choreographies, questa tesi tratta nello specifico i \emph{Choreography Automata}, basati su \emph{Finite State Automata} (FSA).
    In questa tesi viene presentato Choreia: un tool di analisi statica che, partendo da un codice sorgente Go, ricava il Choreography Automata del sisitema concorrente in maniera bottom-up.
\end{abstract}

% Auto-generated table of contents
\tableofcontents

% Introduction chapter
\chapter{Introduzione} % TODO Should be improved
TODO \iffalse
    Negli ultimi anni si è visto un interesse sempre più pronunciato verso lo sviluppo di sistemi concorrenti e distribuiti, questo si riflette nella progettazione a microservizi o in linguaggi di programmazione più recenti come Go, Rust o C++ 20 i quali implementano tutti, in forme disparate, un approccio alla concorrenza facilitato e univoco evitando la complessità e frammentazione che può derivare dall'utilizzo di librerie esterne e non standardizzate.\\
    % ? Keep this paragraph
    Prendendo per esempio Go, talvolta chiamato anche golang, un linguaggio di programmazione creato nel 2009 da Robert Griesemer, Rob Pike e Ken Thompson, che negli anni successivi prende una forte trazione, fino a essere supportato da Google, sopratutto nel settore del web-development e system programming. Alcuni dei fattori da successo sono una compilazione ed esecuzione veloce, una grammatica semplice e snella e un approccio \emph{built-in} alla concorrenza con tipi e primitive fornite direttamente dalla standard library. Quest'ultima infatti fornisce un solo tipo di dato \emph{chan} che può essere \emph{buffered} o \emph{unbuffered}, la comunicazione che avviene su questi canali è rispettivamente asincrona e sincrona.\\
    Parallelamente a questo rinnovato interesse verso la programmazione concorrente e distribuita nasce l'esigenza di formalizzare dei modelli teorici che possano supportare la progettazione, gestione e manutenzione di sistemi concorrenti. Uno di questi modelli sono le \emph{Choreographies}, le quali facilitano la rappresentazione e descrizione di sistemi concorrenti in cui i singoli \emph{attori} comunicano tra di loro. Un aspetto distintivo delle Choreographies è la coesistenza di due \emph{view}:
    \begin{itemize}
        \item Vista \textbf{globale}: descrive la coordinazione necessaria tra i componenti di un sistema
        \item Vista \textbf{locale}: descrive il comportamento di un singolo componente \emph{in isolamento}
    \end{itemize}
    % ? Add the parts about Choreography Automatas properties and how them can help making assumption on the Go program
    L'obiettivo di questa tesi è quello di progettare e implementare un tool che permetta, preso un sorgente Go e per mezzo di analisi statica, di estrarre un Choreography Automata, ovvero un automa a stati finiti che rappresenta la Choreography in modo grafico, più chiaro e immediato rispetto alla lettura del codice sorgente.
\fi


% "Preliminaries and Notation" chapter
% Preliminaries and Notation chapter
\chapter{Nozioni preliminari e notazione}
\label{cap:Preliminaries}
% Finite State Automata (+ related) section
\section{FSA non deterministici e deterministici}
Prima di introdurre le coreografie e i Choreography Automata è necessario fare un breve richiamo di alcune nozioni fondamentali quali la nozione di Automa a Stati Finiti (FSA)\cite{Linguaggi_di_Prorgammazione} e alcune operazioni possibili sugli stessi.
Gli automi a stati finiti sono la descrizione di un sistema dinamico che si evolve nel tempo, esiste un parallelo tra gli automi e i calcolatori moderni, per esempio il flusso d'esecuzione di un programma può essere rappresentato attraverso un automa.
Alcune applicazioni pratiche di questi automi possono essere, per esempio, regular expression (RegEx o RegExp), lexer e parser ma possono essere impiegati, come vedremo in questa tesi, anche nel campo dei sistemi concorrenti.\\
Si noti che sebbene per gli scopi di questa tesi gli automi a stati finiti siano dei costrutti sufficientemente potenti esistono tuttavia altre classi di automi, espressivamente più potenti, ai quali corrispondono altrettante classi di linguaggi (si veda, per esempio, gli automi a pila) tuttavia gli automi appartenenti a questa classe sono tra i più semplici e immediati.

\begin{definition}[Finite State Automata]
    Un automa a stati finiti (FSA) è una tupla A = $\langle \mathcal{S}, s_0, \mathcal{F}, \mathcal{L}, \delta \rangle$ dove:
    \begin{itemize}
        \item $\mathcal{S}$ è un insieme finito di stati
        \item $s_0 \in \mathcal{S}$ è lo stato iniziale dell'automa
        \item $\mathcal{F}$ è l'insieme degli stati finali o di accettazione ($\mathcal{F} \subseteq \mathcal{S}$)
        \item $\mathcal{L}$ è l'alfabeto finito, talvolta detto anche insieme di label ($\epsilon \notin \mathcal{L}$)
        \item $\delta : \mathcal{S} \times (L \cup \{\epsilon\}) \rightarrow \mathcal{P}(\mathcal{S})$ è la funzione di transizione ($\epsilon$ denota la stringa vuota)
    \end{itemize}
\end{definition}

\begin{remark}
    Tipicamente è solito trovare una definizione in cui è presente anche l'insieme degli stati di terminazione (o di accettazione) $\mathcal{F}$. In questo caso non è stato definito e pertanto assumiamo che ogni $s \in \mathcal{S}$ sia uno stato di accettazione.
\end{remark}

\begin{remark}
    Va notato anche che questa definizione coincide con quella di automa a stati finiti \emph{non deterministico}, solitamente indicato in letteratura con la sigla NFA (\emph{Non Deterministic Finite Automata}). Una sottoclasse particolarmente rilevante è quella dei DFA (\emph{Deterministic Finite Automata}) che andremo a definire di seguito.
\end{remark}

\begin{definition}[Deterministic Finite Automata]
    Un automa a stati finiti deterministico è una tupla D = $\langle \mathcal{S}, s_0, \mathcal{F}, \mathcal{L}, \delta \rangle$ dove $\delta : \mathcal{S} \times L \rightarrow \mathcal{S}$
\end{definition}
Le varianti deterministiche si distinguono dalle loro controparti non deterministiche dal fatto che non ammettono né l'utilizzo di $\epsilon$ transizioni, né l'utilizzo di transizioni \emph{uscenti}, dallo stesso stato, con la medesima etichetta. Sebbene queste due varianti siano tra loro equivalenti, l'utilizzo di una variante rispetto all'altra può essere determinato da fattori come: necessità di una maggiore elasticità (gli NFA sono meno stringenti rispetto ai DFA) o di una migliore chiarezza (i DFA sono più immediati e semplici).\\
In ogni caso è sempre possibile, dato un NFA qualunque, ottenere un DFA ad esso equivalente anche se quest'ultimo spesso ha un numero maggiore di stati rispetto all' NFA di partenza. L'algoritmo che permette di fare questa trasformazione fa uso estensivo di \emph{$\epsilon$ closure}\cite{Linguaggi_di_Prorgammazione} e della funzione \emph{mossa}\cite{Linguaggi_di_Prorgammazione} che andremo a definire di seguito:

\begin{definition}[$\epsilon$ closure]
    Fissato un NFA N = $\langle \mathcal{S}, s_0, \mathcal{F}, \mathcal{L}, \delta \rangle$ ed uno stato $s \in \mathcal{S}$ si dice $\epsilon$ closure di s, indicata con $\epsilon$-clos(s), il più piccolo $\mathcal{R} \subseteq \mathcal{S}$ tale che:
    \begin{itemize}
        \item $s \in \epsilon$-clos(s)
        \item se x $\in \epsilon$-clos(s) allora $\delta(x, \epsilon) \subseteq \epsilon$-clos(s)
    \end{itemize}
\end{definition}

\begin{remark}
    Se $\mathcal{X}$ è un insieme di stati definiamo $\epsilon$-clos($\mathcal{X}$) come $\bigcup_{x \in \mathcal{X}} \epsilon \mbox{-clos(x)}$.
\end{remark}

\begin{definition}[Mossa]
    Dato un insieme di stati $\mathcal{X} \subseteq \mathcal{S}$ e un simbolo $\alpha \in \mathcal{L}$ definiamo la funzione \emph{mossa}: $\mathcal{P(S)} \times \mathcal{L} \longrightarrow \mathcal{P(S)}$ tale che: \emph{mossa}$(\mathcal{X}, \alpha) = \bigcup_{x \in \mathcal{X}}\delta(x, \alpha)$, ovvero l'insieme di stati raggiungibili da un dato insieme di stati di partenza, leggendo in input $\alpha$.
\end{definition}
\newpage % TODO Avoid forcefully break page
L'algoritmo che permette di ricavare un DFA da un qualsiasi NFA è il seguente:
\begin{algorithm}
    \caption{Costruzione per sottoinsiemi}\label{alg:SubsetConstruction_Algorithm}
    \begin{algorithmic}
        \State $x \gets$ $\epsilon$-clos($s_0$) \Comment{Lo stato iniziale del DFA}
        \State $\mathcal{T} \gets \{ x \}$                   \Comment{Un insieme di $\epsilon$-clos}
        \While{$\exists$ t $\in \mathcal{T}$ non marcato}
        \State marca(t)
        \ForEach {$\alpha \in \mathcal{L} $}
        \State $r \gets \epsilon$-clos(mossa(t, $\alpha$))
        \If {$r \notin \mathcal{T}$}
        \State $\mathcal{T} \gets \mathcal{T} \cup \{r\}$
        \EndIf
        \State $\delta(t, \alpha) \gets r$  \Comment{Denota che la $\delta$ del DFA con input t ed $\alpha$ darà output r}
        \EndFor
        \EndWhile
    \end{algorithmic}
\end{algorithm}\\
Si noti che $x$, $\mathcal{T}$ e $\delta$ saranno rispettivamente lo stato iniziale, l'insieme degli stati e la funzione di transizione del DFA corrispondente, $\mathcal{F}$ sarà invece l'insieme di tutti i $t \in \mathcal{T}$ che al loro interno contengono almeno uno stato finale dell'NFA di partenza mentre $\mathcal{L}$ rimane invariato. Quindi il DFA ottenuto in output sarà D = $\langle \mathcal{T}, x, \mathcal{F}, \mathcal{L}, \delta \rangle$.

\subsection{Minimizzazione}
Nell'ambito della teoria degli automi esistono una serie di operazioni e trasformazioni che è possibile effettuare, per esempio la composizione di più automi, tuttavia nel nostro caso poniamo particolare riguardo alla minimizzazione. Capita spesso infatti che un automa abbia un numero di stati maggiore del necessario e che alcuni di questi stati siano equivalenti tra loro (e dunque duplicati). Attraverso la minimizzazione è possibile \emph{fondere} insieme questi stati tra loro ottenendo infine un automa più snello (in numero di stati e transizioni) e più facile da comprendere. Si noti questo problema degli stati duplicati non sorge solo dalla progettazione umana ma può anche essere un \emph{side effect} di algoritmi come quello di Costruzione per sottoinsiemi mostrato sopra.\\
L'algoritmo più conosciuto per minimizzare un automa è detto \emph{Algoritmo di Riempimento a Scala}\cite{Linguaggi_di_Prorgammazione} e, di seguito, vedremo il suo funzionamento. Tuttavia occorre fare un'importante premessa prima di introdurre l'algoritmo, il funzionamento dello stesso è legato al fatto che la funzione di transizione $\delta$ sia definita su ogni $\alpha \in \mathcal{L}$, la letteratura distingue gli automi \emph{incompleti}, che non verificano questa condizione, da quelli \emph{completi}.\\
Negli automi incompleti la funzione di transizione è parziale e dunque sorgono dei problemi nel momento in cui cerchiamo di minimizzarli, una soluzione molto semplice è quella di usare uno \emph{stato di errore} (detto anche \emph{stato pozzo}). Essenzialmente si va a completare la funzione di transizione nei casi mancanti (non definiti) con una transizione verso questo stato di errore, allo stesso tempo tutte le transizioni uscenti da questo stato di errore tornano sullo stesso ($\forall_{ \alpha \in \mathcal{L}} \delta(E, \alpha) = E$) il nome di stato di pozzo deriva infatti dal fatto che una volta raggiunto non è possibile uscirne.\\
L'intuizione alla base dell'algoritmo di riempimento a scala è la seguente, valutiamo le singole coppie $(p, q)$ con $p, q \in \mathcal{S}$ e cerchiamo un $\alpha \in \mathcal{L}$ tale che lo stato p si comporti diversamente rispetto allo stato q, questo ci permette di dimostrare che p e q non sono equivalenti e dunque non hanno ragione di essere fusi insieme. Alla fine dell'esecuzione tutte le coppie di stati che non saranno distinte tra loro indicheranno degli stati equivalenti.\\
L'algoritmo di Riempimento della Tabella a Scala è definito come segue:
\begin{algorithm}
    \caption{Riempimento della Tabella a Scala}\label{alg:Minimization_Algorithm}
    \begin{algorithmic}
        \State Inizializza la tabella a scala con le coppie (p,q)
        \State Marca le coppie (x,y) con marca $x_0$ con $x \in \mathcal{F}$ e $y \notin \mathcal{F}$
        \While{$\exists$ almeno un marchio $x_i$ all'iterazione i}
        \If{$\exists \alpha \in \mathcal{L}, \exists p, q \in \mathcal{S}$ tale che $\delta(p, \alpha) \neq \delta(q, \alpha)$}
        \State Marca (p, q) con marca $x_i$
        \EndIf
        \State Considera all'iterazione seguente solo gli stati non marcati
        \EndWhile
    \end{algorithmic}
\end{algorithm}

\subsection{Prodotto}
L'ultima operazione su automi a stati finiti che introduciamo è il \emph{prodotto} tra automi. Solitamente questa operazione viene usata per ricavare, a partire da due o più linguaggi e rispettivi automi, un automa che riconosca l'unione e/o l'intersezione di tali linguaggi.
\begin{definition}[Prodotto di automi]
    \label{def:FSA_Product}
    Siano $A_1 = \langle \mathcal{S}_1, s_{01}, \mathcal{L}_1, \delta_1 \rangle$ e $A_2 = \langle \mathcal{S}_2, s_{02}, \mathcal{L}_2, \delta_2 \rangle$ due automi a stati finiti, il loro prodotto $C = \langle \mathcal{S}, s_0, \mathcal{L}, \delta \rangle$ è definito come segue:
    \begin{itemize}
        \item   $\mathcal{S} = \mathcal{S}_1 \times \mathcal{S}_2$
        \item $s_0 = (s_{01}, s_{02})$
        \item $\mathcal{L} = \mathcal{L}_1 \cup \mathcal{L}_2$
        \item $\delta$ è invece definita nel seguente modo:
    \end{itemize}
    \begin{equation*}
        \begin{cases}
            \text{$\delta ((s_1, s_2), a) = \{ (x,y) | x \in \delta(s_1, a) \land y \in \delta(s_2, a) \}$ \hfill se $\delta_1$ e $\delta_2$ sono definite} \\
            \text{non definito} \hfill \text{altrimenti}
        \end{cases}
    \end{equation*}
\end{definition}
\begin{remark}
    Per quanto riguarda $\mathcal{F}$ si può procedere in due modi diversi: se siamo interessati all'unione dei due linguaggi allora $\mathcal{F} = \{(x,y) | x \in \mathcal{F}_1 \lor y \in \mathcal{F}_2 \} $ mentre se siamo interessati all'intersezione di tali linguaggi prenderemo $\mathcal{F} = \{(x,y) | x \in \mathcal{F}_1 \land y \in \mathcal{F}_2 \}$.
\end{remark}

\subsection{Esempi}
Per concludere questa sezione mostriamo di seguito esempi dei vari concetti definiti in precedenza. Il seguente è un NFA N un grado di riconoscere la Regular Expression $(a|b)^*ba$:
\begin{figure}[ht]
    \centering
    \begin{tikzpicture}[->,>=stealth',node distance=2cm,scale=1, transform shape]
        \node[state,initial] (q0) {$q_0$};
        \node[state] (q1) [right of=q0] {$q_1$};
        \node[state] (q2) [above right of=q1] {$q_2$};
        \node[state] (q3) [right of=q2] {$q_3$};
        \node[state] (q4) [below right of=q1] {$q_4$};
        \node[state] (q5) [right of=q4] {$q_5$};
        \node[state] (q6) [above right of=q5] {$q_6$};
        \node[state] (q7) [right of=q6] {$q_7$};
        \node[state] (q8) [right of=q7] {$q_8$};
        \node[state, accepting] (q9) [right of=q8] {$q_9$};
        \draw
        (q0) edge[above] node{$\epsilon$} (q1)
        (q0) edge[bend left=60, above] node{$\epsilon$} (q7)
        (q1) edge[above] node{$\epsilon$} (q2)
        (q2) edge[above] node{$a$} (q3)
        (q3) edge[above] node{$\epsilon$} (q6)
        (q1) edge[above] node{$\epsilon$} (q4)
        (q4) edge[above] node{$b$} (q5)
        (q5) edge[above] node{$\epsilon$} (q6)
        (q6) edge[above] node{$\epsilon$} (q7)
        (q6) edge[above] node{$\epsilon$} (q1)
        (q7) edge[above] node{$b$} (q8)
        (q8) edge[above] node{$a$} (q9);
    \end{tikzpicture}
    \caption{Un possibile NFA che riconosce la RegEx $(a|b)^*ba$}
    \label{fig:NonDeterministic_Example}
\end{figure}\\
Si noti che questo è solo un \emph{possibile} NFA in grado di riconoscere il linguaggio dato ma ne esistono infiniti altri equivalenti ad esso. Vediamo ora invece il DFA D, equivalente ad N, calcolato tramite l'algoritmo di \emph{Costruzione per sottoinsiemi}
\begin{figure}[ht]
    \centering
    \begin{tikzpicture}[->,>=stealth',shorten >=1pt,auto,node distance=3cm,scale=1, transform shape, baseline=0]
        \node[state,initial] (A) {$A$};
        \node[state] (B) [above right of=A] {$B$};
        \node[state] (C) [right of=A] {$C$};
        \node[state, accepting] (D) [above right of=C] {$D$};
        \draw
        (A) edge[above] node{$a$} (B)
        (A) edge[above] node{$b$} (C)
        (B) edge[loop above] node{$a$} (B)
        (B) edge[above] node{$b$} (C)
        (C) edge[loop below] node{$b$} (C)
        (C) edge[bend right, above] node{$a$} (D)
        (D) edge[above] node{$a$} (B)
        (D) edge[bend right,above] node{$b$} (C);
    \end{tikzpicture}
    \parbox{5cm}{
        \begin{tabular}{c c}
            Stati NFA           & Stati DFA \\
            \{ 0,1,2,4,7 \}     & A         \\
            \{ 1,2,3,4,6,7 \}   & B         \\
            \{ 1,2,4,5,6,7,8 \} & C         \\
            \{ 1,2,3,4,6,7,9 \} & D
        \end{tabular}
    }
    \caption{Il DFA equivalente a quello in figura \ref{fig:NonDeterministic_Example}}
    \label{fig:Deterministic_Example}
\end{figure}\\
Ora andiamo a minimizzare il DFA ottenuto precedentemente rimuovendo gli stati equivalenti tramite l'algoritmo di \emph{Riempimento della Tabella a Scala}
\begin{figure}[ht]
    \centering
    \begin{tikzpicture}[->,>=stealth',shorten >=1pt,auto,node distance=3cm,scale=1, transform shape, baseline=0]
        \node[state,initial] (AB) {$A,B$};
        \node[state] (C) [right of=A] {$C$};
        \node[state, accepting] (D) [right of=C] {$D$};
        \draw
        (AB) edge[loop above] node{$a$} (AB)
        (AB) edge[above] node{$b$} (C)
        (C) edge[loop above] node{$b$} (C)
        (C) edge[bend left, above] node{$a$} (D)
        (D) edge[bend left, above] node{$b$} (C)
        (D) edge[bend left, below] node{$b$} (A);
    \end{tikzpicture}
    \space \space \space
    \parbox{5cm}{
        \begin{tabular}{ | c | c | c | c |}
            \hline B &       & $//$  & $//$  \\
            \hline C & $x_1$ & $x_1$ & $//$  \\
            \hline D & $x_0$ & $x_0$ & $x_0$ \\
            \hline   & A     & B     & C     \\
            \hline
        \end{tabular}
    }
    \caption{Il DFA minimizzato ottenuto da quello in figura \ref{fig:Deterministic_Example}}
    \label{fig:Minimization_Example}
\end{figure}\\
Concludiamo mostrando un esempio di prodotto tra due automi, prendiamo in considerazione i due automi di partenza:
\begin{figure}[ht]
    \centering
    \begin{tikzpicture}[->,>=stealth',shorten >=1pt,auto,node distance=3cm,scale=1, transform shape, baseline=0, initial text=$ $,]
        % Automata 1
        \node[state, initial] (0) {$0$};
        \node[state] (1) [right of=0] {$1$};
        \node[state] (2) [right of=1] {$2$};
        % Automata 2
        \node[state, initial] (3) [right of=2] {$0$};
        \node[state] (4) [right of=3] {$1$};
        % Product automata
        \node[state, initial, initial where=below] (5) [below of=0, right of=1] {$0, 0$};
        \node[state] (6) [below of=0, right of=0] {$0, 1$};
        \node[state] (7) [below of=0, left of=4] {$0, 2$};
        \node[state] (9) [below of=6] {$1, 0$};
        \node[state] (8) [below of=5] {$1, 1$};
        \node[state] (10) [below of=7] {$1, 2$};

        \draw
        % Automata 1
        (0) edge[bend left, above] node{$a$} (1)
        (1) edge[bend left, above] node{$b$} (2)
        (2) edge[bend left, below] node{$c$} (0)
        % Automata 2
        (3) edge[bend left, above] node{$a$} (4)
        (4) edge[bend left, below] node{$a$} (3)
        % Product Automata
        (5) edge node{$a$} (6)
        (5) edge[bend left] node{$a$} (8)
        (6) edge[bend left, above] node{$b$} (7)
        (6) edge[bend left] node{$a$} (9)
        (7) edge[bend left] node{$a$} (10)
        (7) edge node{$c$} (5)
        (8) edge[bend left] node{$b$} (5)
        (8) edge node{$a$} (9)
        (9) edge[bend left] node{$b$} (6)
        (9) edge[bend right, below] node{$b$} (10)
        (10) edge[bend left] node{$b$} (7)
        (10) edge node{$c$} (8);
    \end{tikzpicture}
    \caption{Prodotto tra due automi}
    \label{}
\end{figure}

\section{Choreography Automata}
Passiamo ora alla definizione dei \emph{Choreography Automata} (CA); iniziamo diversificando la nozione di \emph{coreografia} e \emph{Choreography Automata}\cite{Choreography_Automata} il primo è un modello logico che permette di specificare le interazioni tra più attori (siano essi processi, programmi, etc.) all'interno di un sistema (concorrente nel nostro caso) mentre i secondi sono invece un'\emph{istanza} possibile per questo modello. In questo caso noi stiamo scegliendo di rappresentare le coreografie tramite degli Automi a Stati Finiti ma questo non esclude altre possibili realizzazioni.\\
Per prima cosa ricordiamo che le coreografie hanno due tipologie di \emph{view} possibili:
\begin{itemize}
    \item \textbf{Global View}: Che descrive il comportamento dei \emph{partecipanti} "as a whole" specificando anche come questi interagiscono tra loro.
    \item \textbf{Local View}: Che descrive il comportamento di un singolo partecipante in \emph{isolamento} rispetto agli altri.
\end{itemize}
La \emph{scelta implementativa} di utilizzare gli FSA è dovuta al fatto che gli stessi, oltre ad essere semplici ma espressivi, permettono di utilizzare loop "nested" ed "entangled" e permettono di sfruttare in maniera molto conveniente i risultati e le nozioni descritti in precedenza. I Choreography Automata sono dunque dei \emph{casi particolari} di automi a stati finiti in cui le transizioni specificano le interazioni tra i vari partecipanti della coreografia.\\
Un esempio di Choreography Automata è visibile nella figura sottostante, la sintassi delle label sulle transizioni è la seguente: \emph{sender}$\rightarrow$\emph{receiver}:\emph{message}.
\begin{figure}[ht]
    \centering
    \begin{tikzpicture}[->,>=stealth',shorten >=1pt,auto,node distance=3.7cm,scale=1, transform shape]
        \node[state,initial] (q0) {$q_0$};
        \node[state] (q1) [right of=q0] {$q_1$};
        \node[state] (q2) [right of=q1] {$q_2$};
        \draw
        (q0) edge[above] node{$A \rightarrow B: tic$} (q1)
        (q1) edge[above] node{$B \rightarrow C: count$} (q2)
        (q2) edge[bend left,above] node[below]{$C \rightarrow A: toc$} (q0);
    \end{tikzpicture}
    \caption{Un esempio di Choreography Automata}
    \label{fig:ChoreographyAutomata_Example}
\end{figure}\\
In questo caso sono rappresentate le interazioni tra gli attori A,B e C, in particolare: A inizia la comunicazione mandando un messaggio \emph{tic} a B, B (dopo aver ricevuto tale messaggio) invia a sua volta \emph{count} a C ed infine C risponde ad A con messaggio \emph{toc}.

\begin{definition}[Choreography Automata]
    \label{def:Choreography_Automata}
    Un Choreography Automata (c-automata) è un $\epsilon$-free FSA con un insieme di label $\mathcal{L}_{int} = \{ A \rightarrow B : m | A \neq B \in \mathcal{P}, m \in \mathcal{M} \}$ dove:
    \begin{itemize}
        \item $\mathcal{P}$ è l'insieme dei partecipanti (per esempio A, B, ecc)
        \item $\mathcal{M}$ è l'insieme dei messaggi che possono essere scambiati (m, n, ecc)
    \end{itemize}
\end{definition}
\begin{remark}
    Anche se nella definizione non sono ammesse $\epsilon$-transizioni una variante non deterministica rimane sempre possibile, come vedremo anche più avanti in questo lavoro.
\end{remark}

\subsection{CFSM e Local Views}
Ora che abbiamo una definizione formale dei Choreography Automata, possiamo concentrarci sull'estrapolazione delle varie view locali a partire dallo stesso. Ricordiamo che le view locali descrivono il comportamento di un singolo partecipante all'interno della coreografia e che sono ottenute attraverso un'operazione di \emph{proiezione} applicata all'intera coreografia (la view globale). Prima di definire però questa operazione di proiezione serve introdurre il concetto di \emph{Communicating Finite-State Machine (CFSM)}. Come il nome suggerisce questo è sempre un modello basato su automi a stati finiti usato specificatamente per la descrizione delle local views. La principale differenza rispetto ai Choreography Automata sta nel fatto che le label sono \emph{direzionali}, ovvero possono essere del tipo "A B ? m" o "A B ! m" per indicare che A riceve (rispettivamente invia) un messaggio m a B.

\begin{definition}[Communicating Finite-State Machine]
    \label{def:CommunicatingFiniteStateMachine}
    Una Communicating Finite State Machine (CFSM) è un FSA $C$ con insieme di labels: \bigskip \\
    \centerline{$\mathcal{L}_{act} = \{$A B ! m, A B ? m $|$ A, B $ \in \mathcal{P},$ m $ \in \mathcal{M}\}$}
    \\ \\
    dove $\mathcal{P}$ e $\mathcal{M}$ sono definiti come in precedenza.
\end{definition}
Dunque il \emph{soggetto} di un'azione in input "A B ? m" è A, lo stesso vale per l'azione di output "A B ! m", indichiamo quindi con $M_a$ la CFSM che ha solo transizioni con soggetto A. Si noti che esiste ed è possibile definire formalmente una funzione \emph{projection} che assegna ad ogni partecipante $p \in \mathcal{P}$ la sua relativa CFSM $M_p$.\\
Ora che abbiamo introdotto tutti i concetti necessari possiamo definire di seguito l'operazione di \emph{Proiezione} su Choreography Automata. \\
Definiamo brevemente la notazione $s_1 \xrightarrow{a} s_{2}$ come abbreviazione per indicare che esiste una transizione da $s_1$ a $s_2$ con label a, formalmente $\exists a \in \mathcal{L}_{act}, s_1, s_2 \in \mathcal{S}. \delta(s_1, a) = s_2$.

\begin{definition}[Proiezione]
    La proiezione su A di una transizione $t = s_1$ $\xrightarrow{a}$ $s_2$ di un Choreography Automata, scritta $t\downarrow_{A}$ è definita come:
    \begin{equation*}
        t \downarrow_{A} =
        \begin{cases}
            s \xrightarrow{\text{A C ! m}} s' & $se $ a = B \rightarrow C : m \wedge B = A       \\
            s \xrightarrow{\text{B A ? m}} s' & $se $ a = B \rightarrow C : m \wedge C = A       \\
            s \xrightarrow{\epsilon} s'       & $se $ a = B \rightarrow C : m \wedge B, C \neq A \\
            s \xrightarrow{\epsilon} s'       & $se $ a = \epsilon
        \end{cases}
    \end{equation*}
\end{definition}
La proiezione di un CA = $\langle \mathcal{S}, s_0, \mathcal{L}_{int}, \delta \rangle$ sul partecipante $p \in \mathcal{P}$, denotata con CA$\downarrow_p$ è ottenuta ricavando in primis l'automa intermedio:
\bigskip \\
\centerline{$A_p = \langle \mathcal{S}, s_0, \mathcal{L}_{act}, \{ s \xrightarrow{t\downarrow_{p}} s' | s \xrightarrow{t} s' \in \delta \} \rangle$}
\\ \\
Tuttavia, come possiamo vedere nella definizione sopra, questo automa intermedio è non deterministico. è dunque necessario rimuovere le eventuali $\epsilon$ transizioni, ottenendone una versione deterministica e successivamente minimizzare quest'ultima. Entrambe le operazioni sono le medesime definite rispettivamente negli algoritmi \ref{alg:SubsetConstruction_Algorithm} e \ref{alg:Minimization_Algorithm}

\begin{figure}[ht]
    \centering
    \begin{tikzpicture}[->,>=stealth',shorten >=1pt,auto,node distance=3.7cm,scale=1, transform shape]
        \node[state,initial] (A0) {0};
        \node[state] (A1) [below of=A0] {1,2};

        \node[state,initial] (B0) [right of=A0] {0,2};
        \node[state] (B1) [below of=B0] {1};

        \node[state,initial] (C0) [right of=B0] {0,1};
        \node[state] (C1) [below of=C0] {2};

        \draw
        (A0) edge[bend left, above] node[rotate=270]{AB! tic} (A1)
        (A1) edge[bend left, above] node[rotate=90]{AC? toc} (A0)

        (B0) edge[bend left, above] node[rotate=270]{BA? tic} (B1)
        (B1) edge[bend left, above] node[rotate=90]{BC! count} (B0)

        (C0) edge[bend left, above] node[rotate=270]{BC? count} (C1)
        (C1) edge[bend left, above] node[rotate=90]{CA! toc} (C0);
    \end{tikzpicture}
    \caption{Le tre view locali estratte dall'automa in figura \ref{fig:ChoreographyAutomata_Example}}
    \label{fig:Projection_Example}
\end{figure}

\subsection{Composizione delle global views}
Esiste anche un'operazione opposta alla proiezione, infatti è possibile \emph{comporre}\cite{CA_Composition} più Choreography Automata in uno unico che rappresenti le interazioni di tutti gli attori presenti. Questo può essere utile per vari motivi: potremmo avere, per esempio, delle \emph{global views locali} (composte da processi, thread o routine) che talvolta comunicano con altre \emph{global views remote} tramite delle \emph{interfacce} (come endpoint REST, WebSocket o connessioni TCP/IP). Da una situazione come questa può nascere l'esigenza di comporre insieme queste global views in una unica per visualizzare le interazioni (locali e non) che intercorrono tra i vari attori.\\
Introduciamo dunque l'operazione di \emph{composizione}, in questo caso assumiamo che gli insiemi dei partecipanti delle varie global views di partenza sia disgiunto in questo modo si evitano ambiguità nel risultato finale. La composizione si ottiene concatenando due operazioni:
\begin{itemize}
    \item \textbf{Prodotto} tra tutte le $n$ global views
    \item \textbf{Sincronizzazione} dell'automa prodotto precedentemente ottenuto
\end{itemize}
L'operazione di prodotto tra automi è stata definita in \ref{def:FSA_Product} e rimane pressochè invariata, l'operazione di \emph{sincronizzazione} è invece più particolare: abbiamo appurato che le global views comunicano tra loro attraverso le interfacce, possiamo considerare quest'ultime come partecipanti alla coreografia con il solo ruolo di fare \emph{forwarding} dei messaggi tra una view e l'altra. Quindi ogni qualvolta che la global view A vorrà mandare un messaggio a B, manderà un messaggio all'interfaccia I, lo stesso vale per B quando vorrà ricevere messaggi da A. La Sincronizzazione mira proprio a \emph{rimpiazzare} le interazioni che avvengono tramite interfacce con interazioni tra attori effettivi.\\\\
L'operazione di sincronizzazione genera un nuovo automa le cui label sono definite come segue:
\begin{equation*}
    \mathcal{S}(A \times B) =
    \begin{cases}
        p \xrightarrow{A \rightarrow B: m} r \hfill \text{ se } \exists p \xrightarrow{A \rightarrow H: m} q, \exists  q \xrightarrow{K \rightarrow B: m} r. (A \neq B) \\
        p \xrightarrow{A \rightarrow B: m} q \hfill \text{ se } A, B\in \mathcal{P}                                                                                     \\
        \text{nessuna transizione} \hfill \text{altrimenti}
    \end{cases}
\end{equation*}
Si noti che come step aggiuntivo alla trasformazione di sopra tutti gli stati non raggiungibili da quello iniziale verrànno rimossi.\\
Vediamo di seguito l'esempio di una composizione tra due automi:
\begin{figure}[ht]
    \centering
    \begin{tikzpicture}[->,>=stealth',node distance=3.5cm,scale=1, transform shape, initial text=$ $,]
        \node[state, initial] (X0) {$0$};
        \node[state] (X1) [right of=X0] {$1$};
        \node[state] (X2) [right of=X1] {$2$};

        \node[state, initial] (Y0) [right of=X2] {$0$};
        \node[state] (Y1) [right of=Y0] {$1$};

        \node[state, initial] (Z0) [below of=X1] {$0$};
        \node[state] (Z1) [right of=Z0] {$1$};
        \node[state] (Z2) [right of=Z1] {$2$};

        \draw
        (X0) edge node[above]{$A \rightarrow H : m$} (X1)
        (X1) edge node[above]{$I \rightarrow A : n$} (X2)
        (X2) edge[bend left, above] node{$A \rightarrow C : p$} (X0)

        (Y0) edge[bend left, above] node{$B \rightarrow J : m$} (Y1)
        (Y1) edge[bend left, above] node[below]{$K \rightarrow B : n$} (Y0)

        (Z0) edge node[above]{$A \rightarrow B : m$} (Z1)
        (Z1) edge node[above]{$B \rightarrow A : n$} (Z2)
        (Z2) edge[bend right, above] node{$A \rightarrow C : p$} (Z0);
    \end{tikzpicture}
    \caption{Un esempio di composizione tra due Choreography Automata}
    \label{fig:NonDeterministic_Example}
\end{figure}\\

\section{Analisi statica e dinamica}
Ora che abbiamo chiarito le nozioni di base per quanto riguarda la Teoria degli Automi e le Coreografie, passiamo ad un'altro aspetto altrettanto importante per i fini di questi tesi. Considerando che l'obiettivo è quello di ottenere un Choreography Automata partendo da un programma Go dobbiamo determinare in che modo è possibile estrarre delle informazioni da tale programma.\\
A questo riguardo ricordiamo che un programma può avere due \emph{formati}:
\begin{itemize}
    \item \textbf{Testuale}: ovvero il codice sorgente, un testo scritto in un linguaggio \emph{human readable} con una specifica \emph{grammatica} e specifici \emph{costrutti} che descrivono, ad alto livello, i passi che devono essere intrapresi durante la computazione. Questo formato è quello più utilizzata dagli umani in quanto più facile da comprendere (ed eventualmente modificare), tuttavia non è comprensibile ai calcolatori che, come sappiamo, lavorano con formati binari.
    \item \textbf{Binario}: il codice macchina (o codice eseguibile) generato dal compilatore (come nel nostro caso) o dall'interprete. Questo formato è difficilmente comprensibile da un umano, ma al contrario è perfettamente comprensibile per una macchina tant'è che può essere \emph{eseguito} dalla stessa.
\end{itemize}
Se consideriamo la definizione di programma come \emph{un insieme di istruzioni per arrivare ad un risultato finale partendo da input forniti} i formati suddetti sono due rappresentazioni equivalenti del medesimo programma e dunque possono essere usati intercambiabilmente e senza alterare la \emph{sostanza} del programma stesso. \bigskip \\
Tornando all'estrazione dei dati da un programma esistono due diverse di tecniche, legate al \emph{formato} del programma stesso:
\begin{itemize}
    \item \textbf{Analisi statica}\cite{Static_Analysis}: questo tipo di analisi viene eseguita sul codice sorgente, estraendo dei dati dallo stesso ma senza compilarlo ne eseguirlo. Questo tipo di analisi non considera e non è in grado di catturare il \emph{contesto d'esecuzione}, ovvero i fattori esterni che possono influenzare l'esecuzione di un programma a \emph{runtime}.
    \item \textbf{Analisi dinamica}\cite{Dynamic_Analysis}: questo tipo di analisi invece viene fatta attraverso la \emph{profilazione} del programma mentre lo stesso esegue, il programma è dunque in un formato binario. La profilazione può avvenire attraverso dei log emessi dal programma stesso oppure attraverso l'utilizzo di un'altro programma (detto \emph{tracer}) che controlla le operazioni eseguite dal programma target (il \emph{tracee})
\end{itemize}
Entrambe le tecniche presentano i rispettivi vantaggi e svantaggi: l'analisi statica permette una visione più completa in tempo più breve poichè osservando il codice sorgente riesce a catturare tutti i possibili percorsi in cui un programma potrebbe entrare, al contrario l'analisi dinamica non permette di avere sempre una visione completa in quanto è limitata ad osservare solo il percorso che l'esecuzione ha preso in quel momento.\\
Un esempio di questo comportamento è dato da un semplice costrutto come l'\texttt{if-then-else}: tramite l'analisi statica è possibile catturare con facilità entrambi i rami mentre tramite l'analisi dinamica è possibile solo osservare un ramo, quello che a runtime verifica la condizione specificata. Per questo specifico aspetto l'analisi dinamica restituisce dei dati \emph{parziali} e non è possibile fare assunzioni sul ramo che non è stato eseguito, tuttavia l'approssimazione di queste informazioni può essere migliorata eseguendo più profilazioni con input diversi. Si noti però che questo non è sufficiente a garantire che le informazioni siano complete ma solo meglio approssimate e il tempo richiesto per completare l'analisi diventa maggiore (proporzionale rispetto al numero di profilazioni).\\
In maniera opposta l'analisi statica non riesce a catturare completamente l'evoluzione del programma osservato nel tempo o l'influenza che fattori esterni quale il \emph{contesto d'esecuzione} abbiamo sullo stesso, questi aspetti sono invece facilmente osservabili attraverso l'analisi dinamica. Anche in questo caso le informazioni restituite dall'analisi statica sono parziali e vanno approssimate, per esempio usando dei valori predefiniti.\bigskip \\
In conclusione, anche se le tecniche mostrate sopra prese singolarmente rappresentano un ottimo strumento per estrarre informazione da un programma, sono fondamentalmente complementari e andrebbero usate in combinazione per ottenere una visione \emph{completa} del programma stesso.

% TODO Add ref to Static Analysis paper

\subsection{Parsing e AST}
Appurato che l'analisi statica estrae i dati dal formato \emph{testuale} del programma, serve capire come è possibile ottenere informazioni dal codice sorgente Go. Il \emph{parsing} è l'operazione che permette di trasformare del codice sorgente in una struttura dati appropriata (l'\emph{Abstract Syntax Tree} o AST\cite{Abstract_Syntax_Tree}) dalla quale è poi possibile ricavare informazioni in maniera semplificata rispetto al dover utilizzare e manipolare la stringa iniziale (il contenuto testuale del file). Questa operazione non viene solo utilizzata nell'analisi statica ma è anche una fase importante del processo di compilazione (o interpretazione) di qualunque linguaggio di programmazione, il compilatore infatti può utilizzare l'AST per ottimizzare il codice sorgente e, in seguito, per generare il codice binario.\\
In generale, l'utilizzo di un AST fornisce vari vantaggi: il principale è quello di avere una struttura dati ben definita e gerarchica. Da questo consegue che è possibile navigare l'AST (in maniera molto simile ad una classica \emph{visita} su alberi) e questo permette di estrarre dati in maniera algoritmica dall'AST stesso. Inoltre, sempre grazie alla struttura gerarchica, è possibile definire delle trasformazioni per la stessa o validare che rispetti certe proprietà.

% "Changes and adaptations" chapter
% "Changes and adaptations" chapter
\chapter{Coreografie per Go}
\section{Obiettivi e problematiche}
L'obiettivo del progetto, ad lato livello, è quello di prendere del codice sorgente \emph{Go} ed estarre in qualche modo il Choreography Automata associato a quel programma e che esprima in particolare come i processi interagiscono tra loro durante l'esecuzione.\\
L'esecuzione del nostro programma si divide in quattro fasi:
\begin{enumerate}
    \item \textbf{Parsing}: Il codice sorgente viene validato e trasformato in un \emph{Abstact Syntax Tree} (AST)
    \item \textbf{Analisi statica}: Viene navigato l'AST estraendo tutte le informazioni necessarie (i metadati) e salvandole in strutture dati appropriate.
    \item \textbf{Estrazione delle local views}: Partendo dai metadati si generano le local views dei vari processi (o attori).
    \item \textbf{Riconcialiazione}: Ottenere partendo dalle view locali un singolo Choreography Automata che rappresenti l'intera coreografia del sistema (view globale).
\end{enumerate}
Questo approccio è chiaramente \emph{Bottom-Up} mentre l'approccio delle definizioni nel capitolo \ref{cap:Preliminaries} è invece \emph{Top-Down}, infatti abbiamo visto come, partendo dal Choreography Automata possiamo ricavare le singole view locali attraverso l'operazione di \emph{Proiezione}.\\
Come accennato sopra si rende necessaria l'implementazione di un'operazione opposta alla proiezione, chiamata \emph{riconciliazione} che permetta di ottenere un view globale a partire dalle sue singole componenti, ovvero le view locali.

\subsection{Limiti dell'analisi statica} \label{subsec:Static_Analysis_Limits}
Per quanto riguarda la fase di estrazione dei metadati, esplicata brevemente sopra, sorge una inconsistenza con la definizione formale di Choreography Automata data in precedenza: l'insieme $\mathcal{M}$ dei messagi non è determinabile in maniera precisa attraverso l'analisi statica.
Questo tipo di analisi viene effettuata infatti utilizzando solo il codice sorgente e ricavando dei dati senza eseguire in alcun modo il codice stesso (per questo motivo è detta \emph{statica}) e nel caso di Go e altri linguaggi non è possibile solo attraverso l'analisi statica ricavare il valore esatto di tutti i messaggi, questo perchè detto valore può essere soggetto a vari tipi di \emph{side effect} durante l'esecuzione, può essere legato a parametri temporali (timestamp o chron) o input forniti dall'utente. Tutti questi \emph{aspetti}  non sono \emph{catturabili}\cite{Static_Analysis} attraverso l'analisi statica e dunque devono essere  gestiti in maniera oppurtuna.\\ \\
L'esempio sottostante mostra un caso di possibile di codice sorgente Go in cui l'analisi statica non riesce a catturare i valori effettivi dei vari messaggi scambiati tra i processi:

\lstinputlisting[language=Go]{Snippets/DynamicAnalysis.go}
Come possiamo vedere il programma mostrato è in realtà alquanto banale il processo \emph{main} genera un intero random che poi invia su un canale precedentemente condiviso con i due processi \emph{worker}, uno dei due processi riceverà questo intero lo incrementerà e poi lo reinvierà su un canale di output con in aggiunta un timestamp del momento dell'invio. Attraverso l'analisi statica non solo non riusciamo a determinare il valore inviato sul canale "in" nè entrambi i valori inviati sul canale "out". \\\\
Una possibile soluzione a questo tipo di problemi può essere quello di utilizzare un tipo di analisi detta dinamica dove essenzialmente si osserva il programma mentre questo esegue e si raccolgono dati di esecuzione a runtime, questo tipo di analisi risolverebbe il problema posto sopra ma allo stesso tempo ne introrubbe alcuni nuovi. L'analisi dinamica presenta per esempio problemi di \emph{incompletezza dei dati} per esempio: prendiamo un costrutto \emph{if-then-else}, sappiamo che a \emph{runtime} l'esecuzione entrerà in uno o nell'altro ramo questo tuttavia significa che saremo in grado di estrarre informazioni solo riguardanti il ramo in cui l'esecuzione si è svolta e non conosceremo nulla di quanto succede nell'altro. Questo problema diventa ancora più evidente per codice che non è \emph{centrale} nel nostro programma, ovvero quel codice che non essendo \emph{core} viene eseguito di rado e solo al sussistere di condizioni particolari.\\\\
La soluzione adottata qui è in realtà molto semplice: prendiamo $\mathcal{M}$ non come l'insieme dei messaggi scambiati tra i partecipante ma come l'\emph{insieme dei tipi} dei messaggi scambiati, i tipi infatti possono essere inferiti e ricavati senza particolari problemi per mezzo di analisi statica e non limitano l'espressività del modello. Nel caso della figura sopra $\mathcal{M}$ sarà definito come segue: $\mathcal{M} = \{ int, payload \}$ e le label nel Choreography Automata associato saranno del tipo $main \xrightarrow{int} worker$ oppure $worker \xrightarrow{payload} main$.

\subsection{Peculiarità di Go}
Visto che il progetto deve gestire del codice sorgente Go è bene considerare delle particolarità del linguaggio in modo da adattare il modello teorico al linguaggio stesso.
%TODO linkt this up with the next phrase
Mentre i canali e il costrutto sintattico "select" non generano particolari problemi o conflitti con il modello teorico attuale lo stesso non si può dire per le Goroutine, in particolare il problema sta nel fatto che le Goroutine siano intrinsicamente \emph{gerarchiche}, ovvero per ogni programma Go viene avviata sempre e solo una Goroutine (quella che esegue la funzione \emph{main}), sarà poi questa durante la sua esecuzione a farne partire altre, le quali a loro volta potranno avviarne altre ancora e cosi via.
Il problema che sorge da questo approccio deriva dal fatto che nella definizione di Choreography Automata si assume in qualche modo che tutti i partecipanti siano già avviati e pronti a comunicare tra loro mentre per i nostri scopi servirebbe invece sapere qaundo e da chi è stata avviata una Goroutine in modo da poter definire quando la sua \emph{local view} diventa rilevante, ci serve determinare il momento esatto in cui una view locale (risp. Goroutine) inizia a interagire con le altre.\\ \\
Per fare questo possiamo estendere la definizione di Choreography Automata e di Communicating Finite-State Machine date riespettivamente in \ref{def:Choreography_Automata} e \ref{def:CommunicatingFiniteStateMachine} come segue:

\begin{definition}[Choreography Automata (Estesa)]
    Un Choreography Automata (c-automata) è un $\epsilon$-free FSA con un insieme di label:\\ \\
    \centerline{$\mathcal{L}_{ext} = \mathcal{L}_{int} \cup \{ A \bigtriangleup B | A, B \in \mathcal{P}\}$}\\ \\
    con $\mathcal{L}_{int}$ e $\mathcal{P}$ definiti come in \ref{def:Choreography_Automata} e $\mathcal{M}$ definito come in \ref{subsec:Static_Analysis_Limits}
\end{definition}

\begin{definition}[Communicating Finite-State Machine (Estesa)]
    Una Communicating Finite State Machine (CFSM) è un FSA $C$ con insieme di labels:
    \\ \\
    \centerline{$\mathcal{L}_{act} = \{$A B ! m, A B ? m, A $\bigtriangleup$ B $|$ A, B $ \in \mathcal{P},$ m $ \in \mathcal{M}\}$}
\end{definition}

\begin{remark}
    Seppur non interessante per gli scopi di questa tesi è possibile adattare la nozione di proiezione in modo che tenga in considerazione di transizione del tipo $A \bigtriangleup B$ con $A, B \in \mathcal{P}$
\end{remark}

\section{Generazione della coreografia}
TODO % TODO


% "Technology stack" chapter
% "Technology stack" chapter
\chapter{Tecnologie e librerie utilizzate}
\section{Go (golang)}
\subsection{Overview}
Go\cite{Golang} (anche chiamato {golang}) è un linguaggio di programmazione \emph{general purpose} open source sviluppato nel 2007 da Robert Griesemer, Rob Pike e Ken Thompson e poi supportato da Google negli anni a seguire. Fortemente ispirato al C presenta una sintassi minimale e molto semplice, Go è \emph{statically typed} e fornisce un \emph{Garbage Collector} lasciando comunque all'utente la possibilità di interagire con i puntatori e allocare dinamicamente la memoria in modo autonomo.\\
Alcuni dei problemi che Go mira a risolvere sono
\begin{itemize}
    \item \textbf{Controllo restrittivo delle dipendenze}: Infatti per evitare di appesantire l'eseguibile finale Go rifiuta di compilare moduli o file dove non tutte le dipendenze importate vengono utilizzate
    \item \textbf{Compilazione più veloce}: Grazie a quanto detto sopra e alla sintassi estremamente semplice e snella il compilatore riesce a diminuire drasticamente il tempo richiesto alla compilazione mantenendo tutti i vantaggi dell'avere le eventuali ottimizzazioni a \emph{compile time}
    \item \textbf{Approccio semplificato alla concorrenza}: Il linguaggio utilizza le Goroutine, dei \emph{processi leggeri}, le quali permettono un approccio semplificato ed accessibile alla programmazione concorrente
\end{itemize}
Altre feature del linguaggio degne di nota sono: il package manager e l'ecosistema di pacchetti totalmente distribuito  e decentralizzato, il numero di moduli e librerie disponibili, e la grande varietà di architetture supportate (comprensive di \emph{microcontroller} e \emph{embedded systems}).\\
Go è stato utilizzato nello sviluppo di tecnologie molto famose e largamente utilizzate come Docker\cite{Docker} e Kubernetes\cite{Kubernetes} e attualmente viene regolarmente utilizzato da grandi aziende quali Google, MongoDB, Dropbox, Netflix, Uber e altri.

\subsection{Costrutti di concorrenza}
Come accennato sopra Go fornisce un approccio semplificato e built-in alla concorrenza e alla gestione della stessa, il linguaggio permette di avviare dei processi leggeri chiamati Goroutine e scambiare messaggi tra quest'ultimi tramite l'utilizzo di \emph{canali}, i quali permettono sia comunicazione \emph{sincrona} che \emph{asincrona}.\\
Introduciamo brevemente i principali costrutti di concorrenza messi a disposizione dal linguaggio:
\begin{itemize}
    \item \textbf{Canali}: Go fornisce un tipo di dato built-in \texttt{chan} su cui è possibile fare operazioni di \emph{send} e \emph{receive}, i canali possono essere \emph{buffered} e \emph{unbuffered}, i primi permettono una comunicazione asincrona (fino al riempimento del buffer) mentre i secondi permettono solo comunicazione sincrona.
    \item \textbf{Goroutine}: è possibile far partire delle Goroutine anteponendo la keyword \texttt{go} ad una qualsiasi function call, questa funzione verrà eseguita in un contesto condiviso (si preservano gli \emph{scope} e le variabili locali) ma parallelo rispetto alla Goroutine che l'ha creato.
    \item \textbf{Select}: Un costrutto particolare che permette di eseguire operazioni di invio o ricezione su più canali ed eseguire la prima, tra queste operazioni, che non sia bloccante, oltre a questo è possibile definire anche un blocco da eseguire una volta completata suddetta operazione. Opzionalmente è possibile definire un blocco di default che viene eseguito quando nessuna delle operazioni sopra può essere completata in maniera non bloccante.
\end{itemize}
Oltre ai costrutti presentati sopra la \emph{standard library} mette a disposizione altri tipi di dato e costrutti \emph{classici} come \emph{Mutex}, \emph{Semafori}, \emph{Monitor} che tuttavia non verrànno trattati in questa tesi.
\newpage % TODO Avoid forcefully break page
\lstinputlisting[language=Go, caption=Esempio di utilizzo dei costrutti di concorrenza forniti da Go]{Snippets/GoConcurrency.go}
\bigskip
Come possiamo vedere in questo esempio: l'esecuzione parte dalla Goroutine \texttt{main} che inizializza i canali \texttt{a} e \texttt{b} e li passa alle Goroutine \texttt{fuzzer}, dopodichè, mentre le due nuove Goroutine inviano 10 messaggi ciascuna (con un timeout tra un invio e l'altro), la Goroutine \texttt{main} attende i vari messaggi tramite la \texttt{select} su entrambi i canali, questo significa che la Goroutine \texttt{main} eseguirà la prima operazione \emph{non bloccante} oppure il \texttt{default} branch se nessun canale ha un messaggio in coda. Il for loop terminerà solamente quando entrambi i canali saranno chiusi dalle rispettive Goroutine \texttt{worker} con l'apposita primitiva \texttt{close}.

\section{Graphviz e DOT}
Vista la necessità di \emph{rappresentare} in qualche modo il Choreography Automata finale e gli eventuali risultati intermedi si è reso necessario l'utilizzo di un qualche tipo di \emph{meccanismo di serializzazione}. Fortunatamente considerando la somiglianza tra Finite State Automata e Grafi (i secondi sono una generalizzazione dei primi) abbiamo potuto riutilizzare tool e strumenti pensati \emph{principalmente} per quest'ultimi.\\
Abbiamo quindi scelto di usare Graphviz\cite{Graphviz}, una libreria open source per la visualizzazione di grafi la quale utilizza DOT\cite{DOT_Lang}, un formato specificatamente progettato per la descrizione dei grafi.\\
La scelta è ricaduta su DOT e Graphviz per alcuni motivi principali:
\begin{itemize}
    \item Il linguaggio DOT è \emph{human readable} e particolarmente facile da comprendere, inoltre Graphviz permette di \emph{convertire} o \emph{esportare} in formati di uso più comune come PNG o SVG
    \item Permette un utilizzo combinato con \emph{Corinne}\cite{Corinne},un tool grafico per la visualizzazione e manipolazione dei Choreography Automata
    \item Essendo Graphviz ormai uno standard \emph{de facto} sono presenti librerie e binding che ne permettono l'utilizzo con moltissimi linguaggi di programmazione, tra cui Go
\end{itemize}
Di seguito un mostriamo un esempio banale di Choreography Automata definito attraverso il linguaggio DOT.
\begin{lstlisting}[caption=Rappresentazione in DOT dell'automa in figura \ref{fig:ChoreographyAutomata_Example}]
    digraph DOT_Graph_Example {
        node [shape=circle, fontsize=20]
        edge [length=100, fontcolor=black]
      
        q0 -> q1[label="A->B:tic"];
        q1 -> q2[label="B->C:count"];
        q2 -> q0[label="C->A:toc"];
    }
\end{lstlisting}
Il seguente esempio definisce un grafo direzionato con tre nodi (q0, q1, e q2) e altrettanti archi, rispettivamente da q0 a q1, da q1 a q2 e da q2 a q0 con le rispettive label. DOT fornisce anche la possibilità all'utente di definire prima tutti i nodi che fanno parte del grafo e poi tutti gli archi.\\
Chiaramente i Choreography Automata generati da Choreia non saranno così semplici e immediati, ciononostante dovrebbe essere comunque possibile interagirvi e comprenderli.

% "Tool description" chapter
% "Tool description" chapter
\chapter{Choreia}
Choreia è il tool sviluppato come progetto per questa tesi. Il tool si occupa di tutte le fasi descritte nel capitolo precedente e consente all'utente finale di esportare il Choreography Automaton ricavato dal codice sorgente in formato DOT. È un software \emph{open source} con licenza GPL-3.0\cite{GPL-3.0} scritto completamente in Go, non richiede alcun tipo di setup se non l'installazione iniziale delle dipendenze. È disponibile al download al seguente url: \url{https://github.com/its-hmny/Choreia}\\

\section{Parametri da linea di comando}
Il tool non ha una GUI in quanto, per gli scopi attuali, non è necessaria: infatti non è stato progettato come un tool di uso comune ma come uno strumento per persone interessate e con un minimo di conoscenza pregressa.\\
In ogni caso è possibile tramite \emph{command line} fornire alcuni parametri e flags per un utilizzo \emph{personalizzato}, i parametri disponibili sono:
\begin{table}[h!]
    \centering
    \begin{tabular}{l l l}
        Breve & Esteso   & Descrizione                                               \\
        -i    & --input  & Il \emph{path} del file .go in input                      \\
        -o    & --output & La directory in cui verranno salvati i file i vari automi \\
        -t    & --trace  & Stampa l'AST sullo \texttt{stdout}                        \\
        -h    & --help   & Mostra un messaggio di aiuto con una breve spiegazione    \\
    \end{tabular}
    \caption{La lista di argomenti da linea di comando  accettati da Choreia}
\end{table}

\section{Struttura del progetto}
Il progetto è strutturato su 3 moduli principali, ognuno con uno specifico compito: Il modulo \texttt{fsa} fornisce un'implementazione per gli automi a stati finiti (implementati attraverso un multigrafo), il modulo \texttt{static analysis} gestisce il parsing (fatto tramite la \emph{standard library} golang) e l'estrazione dei metadati dall'AST ed infine il modulo \texttt{transforms} implementa varie operazioni su FSA come per esempio: determinizzazione e minimizzazione (per FSA generici) o la composizione (per automi associati alle local views). \\
La validazione dei dati e l'orchestrazione delle funzionalità fornite dai vari moduli sono gestite nel \texttt{main} che agisce come entry point del programma.\bigskip \\
\begin{forest}
    for tree={
    font=\ttfamily,
    grow'=0,
    child anchor=west,
    parent anchor=south,
    anchor=west,
    calign=first,
    inner xsep=7pt,
    edge path={
            \noexpand\path [draw, \forestoption{edge}]
            (!u.south west) +(7.5pt,0) |- (.child anchor) pic {folder} \forestoption{edge label};
        },
    % style for your file node 
    file/.style={edge path={
                    \noexpand\path [draw, \forestoption{edge}]
                    (!u.south west) +(7.5pt,0) |- (.child anchor) \forestoption{edge label};},
            inner xsep=2pt,font=\small\ttfamily
        },
    before typesetting nodes={
            if n=1
                {insert before={[,phantom]}}
                {}
        },
    fit=band,
    before computing xy={l=15pt},
    }
    [Choreia
        [cmd
                [main.go, file]
        ]
        [go.mod, file]
        [go.sum, file]
        [internal
                [fsa
                        [\text{fsa.go, transition.go}, file]
                ]
                [static analysis
                        [\text{branch.go, channel.go, function.go, loop.go, ...}, file]
                ]
                [transforms
                        [\text{extraction.go, minimization.go, composition.go, ...}, file]
                ]
        ]
        [LICENSE, file]
        [README, file]
    ]
\end{forest}

\section{Flusso d'esecuzione}
Il flusso d'esecuzione riprende le quattro macro fasi definite nella sezione \ref{sec:GoOutline}.\bigskip\\
Come prima cosa il file in input viene validato e parsato, generando un AST (queste funzionalità sono fornite dal modulo \texttt{go/ast} della standard library), l'AST viene poi navigato da un \emph{visitor} che si occupa di raccogliere e organizzare i dati in apposite strutture dati. Durante questa fase vengono generati degli automi a stati finiti non deterministici associati ad ogni funzione dichiarata all'interno del file in input, questi NFA descriveranno l'evoluzione della computazione nello scope di funzione.\bigskip\\
Una volta ottenuti gli automi e i metadati si procede con lo step successivo, ovvero ricavare gli automi associati alle Goroutine presenti nel programma. Per fare questo si utilizza l'algoritmo \ref{alg:Local_Views_Extraction} che attraverso l'\emph{inlining} delle chiamate di funzione e la sostituzione dei parametri formali con quelli attuali ricava in maniera ricorsiva degli automi non deterministici che rappresentano il flusso d'esecuzione completo della Goroutine. Come fase finale di questo step gli automi vengono determinizzati e minimizzati tramite gli algoritmi mostrati nel capitolo \ref{cap:Preliminaries}.\bigskip\\
L'ultimo step dell'esecuzione è quello di generare la global view a partire dalle view locali (e rispettivi automi), esattamente come detto nella sezione \ref{sec:Go_Composition} generiamo l'automa prodotto di tutte le local views e poi eseguiamo l'operazione di sincronizzazione su quest'ultimo, infine esportiamo questo automa \emph{finale} in formato DOT.

\section{Esempi pratici}
Di seguito mostriamo alcuni esempi di programmi Go e il rispettivo Choreography Automaton associatogli da Choreia. % TODO add link to Choreia repo /folders


\subsection{Loop determinato su canale}
In questo esempio possiamo vedere che la Goroutine \texttt{main} inizializza il canale \texttt{channel} ed avvia la Goroutine \texttt{worker}, quest'ultima invia 100 messaggi numerati sul canale condiviso concludendo con la chiusura dello stesso. La chiusura di un canale è importante poichè blocca ogni futuro utilizzo dello stesso da entrambi i lati della comunicazione e permette al costrutto \texttt{for range} di interrompere il loop una volta ricevuti tutti i messaggi in coda. Se \texttt{worker} non chiudesse il canale \texttt{main} si bloccherebbe, alla 101-esima iterazione, aspettandosi di ricevere un messaggio da un canale \emph{abbandonato} e questo creerebbe un problema di \emph{liveness}.
\newpage % TODO avoid forcefully breaking page
\lstinputlisting[language=Go]{Snippets/ForLoop.go}
\bigskip
Di seguito troviamo gli automi delle local views associate rispettivamente alle Goroutine $\texttt{main}_0$ e $\texttt{worker}_1$ ottenuti tramite l'algoritmo di estrazione delle local views. Come si può facilmente osservare rispecchiano perfettamente il flusso d'esecuzione descritto dal codice che in questo caso è particolarmente semplice.\\
Si noti che l'automa della Goroutine $\texttt{worker}_1$ non presenta una transizione del tipo $\rightarrow shared$, come si potrebbe essere indotti a pensare, bensì la label della transizione è $  \rightarrow channel$ poichè l'algoritmo di estrazione gestisce la sostituzione dei parametri formali con quelli attuali.
\begin{figure}[h!]
    \centering
    \begin{tikzpicture}[->,>=stealth',shorten >=1pt,auto,node distance=3cm,scale=1, transform shape, baseline=0]
        \node[state,initial, initial text=$\texttt{main}_0$] (main0) {$0$};
        \node[state, accepting] (main1)[right of=main0] {$1$};
        \node[state, initial, initial text=$\texttt{worker}_1$, accepting] (worker0)[right=9.5cm] {$0$};
        \draw
        (main0) edge node{$\bigtriangleup worker_1$} (main1)
        (main1) edge [loop right] node{$\leftarrow channel$} (main1)
        (worker0) edge [loop right] node{$\rightarrow channel$} (worker0);
    \end{tikzpicture}
\end{figure} \bigskip \\
Il seguente è invece l'automa associato alla global view ottenuto con l'algoritmo di composizione a partire dai due precedenti. Chiaramente in questo caso la sincronizzazione è banale e avviene sulle sole due transizione di invio e ricezione presenti.
\begin{figure}[h!]
    \centering
    \begin{tikzpicture}[->,>=stealth',shorten >=1pt,auto,node distance=4.5cm,scale=1, transform shape, baseline=0]
        \node[state,initial] (0){$0$};
        \node[state, accepting] (1)[right of=0] {$1$};
        \draw
        (0) edge node{$main_0 \bigtriangleup worker_1$} (1)
        (1) edge [loop right] node{$main_0 \leftarrow worker_1$} (1);
    \end{tikzpicture}
\end{figure}


\subsection{Operazioni condizionali con \texttt{select}}
In questo esempio la Goroutine \texttt{main} crea due canali \texttt{chanA} e \texttt{chanB} e avvia due Goroutine \texttt{responder}, ciascuna con uno dei due canali, dopodichè tramite il costrutto \texttt{select} esegue il primo ramo che presenta un'operazione \emph{non bloccante}, nel caso in cui nessuno dei due canali abbia dei messaggi si mette \emph{in ascolto} aspettando che uno dei due abbia almeno un messaggio in coda. Nel nostro caso le Goroutine \texttt{responder} inviano solo un messaggio e non eseguono altre operazioni tuttavia possiamo assumere che portino avanti computazioni più complesse, inviando alla fine il risultato delle stesse sul canale.\\
\lstinputlisting[language=Go]{Snippets/Select.go}
Gli automi associati alle Goroutine $\texttt{main}_0$, $\texttt{responder}_1$, $\texttt{responder}_2$ sono i seguenti. Si noti come viene \emph{mappato} il costrutto \texttt{select} attraverso gli automi, dal momento che il costrutto permette di \emph{valutare} operazioni su più canali eseguendo solo la prima tra queste che non sia bloccante. Nell'automa a stati finiti questo viene mappato come una scelta non deterministica tra due flussi di esecuzione che poi procedono nei rispettivi \emph{sottografi} ed, eventualmente, si ricongiungono tra loro a fine esecuzione.\bigskip \\
\begin{figure}[h!]
    \centering
    \begin{tikzpicture}[->,>=stealth',shorten >=1pt,auto,node distance=3.5cm,scale=0.9, transform shape, baseline=0]
        \node[state, initial, initial text=$\texttt{main}_0$] (main0) {$0$};
        \node[state] (main1)[right of=main0] {$1$};
        \node[state] (main2)[right of=main1] {$2$};
        \node[state] (main3)[right of=main2, above=0.3cm] {$3$};
        \node[state, accepting] (main4)[right of=main3] {$4$};
        \node[state] (main5)[right of=main2, below=0.3cm] {$5$};
        \node[state, accepting] (main6)[right of=main5] {$6$};

        \node[state, initial, initial where=above, initial text=$\texttt{responder}_1$] (responder10)[below of=main0, right=0.7cm] {$0$};
        \node[state] (responder11)[right of=responder10] {$1$};

        \node[state, initial, initial where=above, initial text=$\texttt{responder}_2$] (responder20)[right of=responder11] {$0$};
        \node[state] (responder21)[right of=responder20] {$1$};
        \draw
        (main0) edge node{$\bigtriangleup responder_1$} (main1)
        (main1) edge node{$\bigtriangleup responder_2$} (main2)
        (main2) edge node[rotate=13, above]{$\leftarrow chanA$} (main3)
        (main3) edge node{$\leftarrow chanB$} (main4)
        (main2) edge node[rotate=-13, below]{$\leftarrow chanB$} (main5)
        (main5) edge node{$\leftarrow chanA$} (main6)

        (responder10) edge node{$\rightarrow chanA$} (responder11)

        (responder20) edge node{$\rightarrow chanB$} (responder21)
        ;
    \end{tikzpicture}
\end{figure}\\
Nel Choreography Automaton generato possiamo vedere come in questo caso l'utilizzo del \emph{prodotto tra automi} permette di considerare tutte le possibili evoluzioni del sistema concorrente che poi verrano elaborate ulteriormente e poste nella forma attuale dalla \emph{sincronizzazione}, che in questo caso genera correttamente entrambi i rami d'esecuzione.\bigskip \\
\begin{figure}[h!]
    \centering
    \begin{tikzpicture}[->,>=stealth',shorten >=1pt,auto,node distance=4.8cm,scale=0.9, transform shape, baseline=0]
        \node[state,initial] (0) {$0$};
        \node[state] (1)[right of=0] {$1$};
        \node[state] (2)[right of=1] {$2$};
        \node[state] (3)[right of=2, above=40pt] {$3$};
        \node[state, accepting] (4)[left of=3, above=40pt] {$4$};
        \node[state] (5)[right of=2, below=40pt] {$5$};
        \node[state, accepting] (6)[left of=5, below=40pt] {$6$};
        \draw
        (0) edge node{$main_0 \bigtriangleup responder_1$} (1)
        (1) edge node{$main_0 \bigtriangleup responder_2$} (2)
        (2) edge node[rotate=22, above]{$main_0 \leftarrow responder_1$} (3)
        (3) edge node[rotate=-22, above]{$main_0 \leftarrow responder_2$} (4)
        (2) edge node[rotate=-22, below]{$main_0 \leftarrow responder_2$} (5)
        (5) edge node[rotate=22, below]{$main_0 \leftarrow responder_1$} (6);
    \end{tikzpicture}
\end{figure}

\subsection{Branching con \texttt{if-then-else}}
In questo esempio la Goroutine \texttt{main} inizializza due canali \texttt{A} e \texttt{B} ed avvia due Goroutine \texttt{dummy} ciascuna con il proprio canale: Una volta completata questa prima fase di setup semplicemente riceve un messaggio da \texttt{A} e uno da \texttt{B}. Si noti che la receive su \texttt{A} è sempre effettuata poichè la condizione dell'\texttt{if} è sempre verificata.
\lstinputlisting[language=Go]{Snippets/If-Then-Else.go}
Gli automi associati alle Goroutine $\texttt{main}_0$, $\texttt{dummy}_1$, $\texttt{dummy}_2$ sono i seguenti. Si noti come viene \emph{mappato} l'\texttt{if-then-else} attraverso gli automi: la biforcazione del ramo then viene mappata correttamente ma allo stesso tempo procede un'esecuzione \emph{lineare} che rappresenta il caso in cui la condizione dell'if non sia verificata. Sempre riguardante la condizione dell'if possiamo banalmente osservare le limitazioni dell'analisi statica descritte al capitolo \ref{cap:Preliminaries}: nonostante la condizione sia sempre verificata Choreia assume del \emph{non determinismo} e biforca il flusso d'esecuzione.
\newpage % TODO avoid forcefully breaking page
\begin{figure}[t!]
    \centering
    \begin{tikzpicture}[->,>=stealth',shorten >=1pt,auto,node distance=3.5cm,scale=0.9, transform shape, baseline=0]
        \node[state,initial, initial text=$\texttt{main}_0$] (main0) {$0$};
        \node[state] (main1)[right of=main0] {$1$};
        \node[state] (main2)[right of=main1] {$2$};
        \node[state] (main3)[above=2cm, right of=main2] {$3$};
        \node[state, accepting] (main4)[below=2cm, right of=main3] {$4$};

        \node[state,initial, initial text=$\texttt{dummy}_1$] (dummy10)[below=2cm] {$0$};
        \node[state, accepting] (dummy11)[right of=dummy10] {$1$};

        \node[state,initial,initial text=$\texttt{dummy}_2$] (dummy20)[right of=dummy11] {$0$};
        \node[state, accepting] (dummy21)[right of=dummy20] {$1$};
        \draw
        (main0) edge node{$\bigtriangleup dummy_1$} (main1)
        (main1) edge node{$\bigtriangleup dummy_2$} (main2)
        (main2) edge node{$\leftarrow B$} (main4)
        (main2) edge node[above, rotate=34]{$\leftarrow A$} (main3)
        (main3) edge node[above, rotate=-32]{$\leftarrow B$} (main4)

        (dummy10) edge node{$\rightarrow A$} (dummy11)

        (dummy20) edge node{$\rightarrow B$} (dummy21);
    \end{tikzpicture}
\end{figure}
Nel Choreography Automaton generato possiamo vedere che la struttura dello stesso è del tutto simile a quella dell'automa associato alla Goroutine $\texttt{main}_0$. Questo è dovuto al fatto che almeno per questi esempi il sistema concorrente è particolarmente semplice e \emph{gerarchico}: il \texttt{main} è la Goroutine dominante mentre le altre Goroutine sono piuttosto banali e immediate, dunque non aggiungono complessità al Choreography Automaton finale. Questi esempi \emph{semplici} sono dovuti alla difficoltà di rappresentazione degli automi e all'incremento nella complessità degli stessi al crescere della complessità del sistema concorrente, tuttavia Choreia è in grado di riconoscere e gestire sistemi concorrenti più complessi ed intricati che sarebbero difficili da riportare graficamente in questa tesi.
\begin{figure}[h!]
    \centering
    \begin{tikzpicture}[->,>=stealth',shorten >=1pt,auto,node distance=4.3cm,scale=0.75, transform shape, baseline=0]
        \node[state,initial] (main0) {$0$};
        \node[state] (main1)[right of=main0] {$1$};
        \node[state] (main2)[right of=main1] {$2$};
        \node[state] (main3)[above=2cm, right of=main2] {$3$};
        \node[state, accepting] (main4)[below=3cm, right=17cm] {$4$};
        \draw
        (main0) edge node{$main_0 \bigtriangleup dummy_1$} (main1)
        (main1) edge node{$main_0 \bigtriangleup dummy_2$} (main2)
        (main2) edge node{$main_0 \rightarrow dummy_2$} (main4)
        (main2) edge node[above, rotate=26]{$main_0 \rightarrow dummy_1$} (main3)
        (main3) edge node[above, rotate=-23]{$main_0 \rightarrow dummy_2$} (main4);
    \end{tikzpicture}
\end{figure}


\subsection{Function call con sostituzione dei parametri}
In questo esempio la Goroutine \texttt{main}, dopo aver creato il canale \texttt{channel} chiama la funzione \texttt{f} passandogli suddetto canale come argomento, questa funzione esegue le seguenti operazioni: invia un messaggio sul canale, avvia la Goroutine \texttt{dummy} condividendo con essa il canale e infine riceve il messaggio inviato precedentemente prima di restituire il controllo a \texttt{main} che conclude ricevendo per la seconda volta dal canale.
\newpage % TODO Avoid forcefully breaking page
\lstinputlisting[language=Go]{Snippets/FunctionCall.go}
Gli automi associati alle Goroutine $\texttt{main}_0$ e $\texttt{dummy}_1$ sono i seguenti. Si noti che la funzione \texttt{f} subisce l'\emph{inlining} durante la fase di estrazione delle local views, difatti possiamo vedere che alcune delle transizioni nel primo automa sono proprio derivanti dall'esecuzione della funzione \texttt{f}.
\begin{figure}[h!]
    \begin{tikzpicture}[->,>=stealth',shorten >=1pt,auto,node distance=3.2cm,scale=1, transform shape, baseline=0]
        \node[state,initial,initial text=$\texttt{main}_0$] (main0) {$0$};
        \node[state] (main1)[right of=main0] {$1$};
        \node[state] (main2)[right of=main1] {$2$};
        \node[state] (main3)[right of=main2] {$3$};
        \node[state, accepting] (main4)[right of=main3] {$4$};

        \node[state,initial,initial text=$\texttt{dummy}_1$] (dummy0)[below of=main1, right=1cm] {$0$};
        \node[state, accepting] (dummy1)[right of=dummy0] {$1$};
        \draw
        (main0) edge node{$\rightarrow channel$} (main1)
        (main1) edge node{$\bigtriangleup dummy$} (main2)
        (main2) edge node{$\leftarrow channel$} (main3)
        (main3) edge node{$\leftarrow channel$} (main4)

        (dummy0) edge node{$\rightarrow channel$} (dummy1);
    \end{tikzpicture}
\end{figure}\\
Choreia, allo stato attuale, non è in grado di generare un automa per il seguente codice, questo è dovuto al fatto che sono presenti diverse criticità all'interno dello stesso: prima di tutto la Goroutine \texttt{main} si sincronizza con se stessa inviando il primo messaggio su \texttt{channel} e ricevendolo poi dallo stesso. Questo, seppur possibile teoricamente in Go, non è \emph{catturabile} dal nostro algoritmo in quanto richiederebbe un adattamento della nozione di prodotto tra automi. Attualmente infatti il prodotto tra automi $A \times B$ non considera il caso con A utilizzato da entrambi i lati come fattore ($A \times A$) e questo comporta una \emph{perdita d'informazione}.\\
Inoltre, anche ammettendo che fosse implementato quanto detto sopra, a questo punto avremmo nel Choreography Automaton ben due sincronizzazione del tipo $main \rightarrow main$ sia nella transizione da 0 a 1, sia nella transizione da 2 a 3 e andrebbe quindi implementato un qualche tipo di algoritmo e/o validazione che eviti di incappare in questi casi particolari.\\
Un'ulteriore problematica legata a questo esempio invece è dovuta al fatto che il nostro algoritmo genererebbe due transizioni da 2 a 3 rispettivamente con label $main \rightarrow main$ e $dummy \rightarrow main$ questo perchè l'algoritmo di composizione attuale non tiene in considerazione che i canali sono gestiti con un politica di \emph{First In First Out (FIFO)} e dunque in questo caso l'unica sincronizzazione possibile tra gli stati 2 e 3 è quella tra main e sè stesso.\bigskip \\
Il Choreography Automaton corretto è il seguente:
\begin{figure}[h!]
    \centering
    \begin{tikzpicture}[->,>=stealth',shorten >=1pt,auto,node distance=4.5cm,scale=0.95,transform shape, baseline=0]
        \node[state,initial] (0) {$0$};
        \node[state] (1)[right of=0] {$1$};
        \node[state] (2)[right of=1] {$2$};
        \node[state, accepting] (3)[right of=2] {$3$};
        \draw
        (0) edge node{$main_0 \bigtriangleup dummy_1$} (1)
        (1) edge node{$main_0 \rightarrow main_0$} (2)
        (2) edge node{$dummy_1 \rightarrow main_0$} (3)
        ;
    \end{tikzpicture}
\end{figure}

% "Conclusion and future works" chapter
% "Conclusion and future works" chapter
\chapter{Conclusioni e lavori futuri}
Complessivamente Choreia soddisfa i requisiti che ci siamo posti e permette di fare quello per cui è stato concepito, tuttavia il tool è un \emph{proof of concept} e dunque ancora grezzo e prototipale. Il tool non è esente da possibili miglioramenti, abbiamo visto infatti che esistono alcuni pattern e costrutti di Go che non sono attualmente supportati così come \emph{edge cases} che portano a problemi non risolvibili con le tecniche adottate attualmente. In ogni caso il tool è open source e distribuito con licenza GPL-3.0 e siamo aperti in futuro a proposte e contribuzioni esterne.\bigskip \\
Alcune delle migliorie principali da apportare nel futuro sono:
\begin{itemize}
    \item \textbf{Migliorare la fase di analisi}: migliorando l'approssimazione degli automi associati alle singole funzioni durante la visita dell'AST possiamo migliorare tutti gli automi successivi che sono generati a partire dagli stessi. Inoltre attualmente è possibile effettuare solo il parsing di singoli file, questo esclude l'utilizzo del tool su casi reali e limita l'utente.
    \item \textbf{Supportare altri costrutti}: abbiamo già spiegato che per i fini di questa tesi non consideriamo altri costrutti di concorrenza forniti dalla standard library di Go come Waitgroup, Lock, etc... tuttavia essendo questi costrutti largamente utilizzati all'interno di progetti \emph{reali} questa \emph{mancanza} limita l'applicabilità del nostro tool ad uno \emph{scenario reale}. Aggiungendo supporto per i costrutti sopracitati potremmo iniziare a confrontare il nostro tool con dei casi d'uso effettivi e reali e delineare un confronto con altri tool simili.
    \item \textbf{Migliorare l'algoritmo di composizione}: l'algoritmo di composizione delle local views è stato introdotto per la prima volta in questa tesi partendo dalla sua controparte per la composizione delle global views \cite{CA_Composition} tuttavia esistono degli \emph{edge cases} in cui l'algoritmo attuale riscontra delle difficoltà e serve dunque valutare questi nuovi aspetti e delineare l'evoluzione futura dello stesso.
    \item \textbf{Performance improvements}: finora il tool è stato utilizzato solo su input relativamente semplici e automi di piccole dimensione, aggiungendo nuove funzionalità e supporto per costrutti più complessi non escludiamo che si possano presentare bottleneck alle performance dovute alla complessità computazionale di certe operazioni.
\end{itemize}
Un obiettivo per il futuro è sicuramente quello di ottenere una versione sufficientemente sofisticata da poter essere messa a confronto con altri tool simili al nostro i quali offrono funzionalità di analisi di sistemi concorrenti generando solitamente report sulla \emph{solidità} dello stesso. Un tool molto interessante sotto questo aspetto e che assomiglia per alcuni aspetti a quanto fatto da noi è Gomela\cite{Gomela}: il tool esattamente come Choreia adotta un approccio bottom-up e considera le singole funzioni come \emph{building blocks} da cui partire. La differenza sta nel fatto che, invece che associare un automa ad ogni funzione, Gomela mappa il flusso d'esecuzione attraverso un \emph{behavioural type language} chiamato Promela\cite{Promela}. Una volta descritto il sistema concorrente tramite Promela, questa descrizione può essere usata da dei \emph{model checker} come Spin\cite{Spin} (Simple Promela Interpreter) che si occupano di verificarne la \emph{correttezza} attraverso la simulazione del sistema concorrente in multipli contesti d'esecuzione.\bigskip \\
Segnaliamo che, al momento della scrittura di questa tesi, è stata rilasciata un versione \emph{non stabile} e \emph{sperimentale} del modulo \texttt{go/ssa}. Questo modulo permette, partendo dall'AST, di estrarre una versione in \emph{Static Single Assignment (SSA) Form}\cite{SSA_Form}, questa \emph{rappresentazione intermedia} è usata dai compilatori e dagli static analyzer in quanto fornisce un \emph{contesto di immutabilità} (ogni variabile viene assegnata esattamente una sola volta) che a sua volta permette di \emph{tracciare} in modo più facile ed efficiente le variabili e i \emph{mutamenti} che queste subiscono durante l'esecuzione come \emph{riassegnamento} e \emph{side-effect}. Inoltre, essendo una rappresentazione intermedia, presenta una grammatica più semplice rispetto all'AST di partenza e può aver subito già alcune ottimizzazioni e \emph{linearizzazioni} durante le fasi precedenti.\\
Una volta rilasciata una versione stabile del modulo \texttt{go/ssa} sarebbe interessante integrarlo all'interno di Choreia al fine di ottenere una migliore approssimazione delle singole funzioni tramite automi a stati finiti, quest'ultimi produrranno a loro volta dei Choreography Automaton meglio approssimati.

\end{document}