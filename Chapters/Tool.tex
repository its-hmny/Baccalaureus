% "Tool description" chapter
\chapter{Choreia}
Choreia è il tool sviluppato come progetto per questa tesi. Il tool si occupa di tutte le fasi descritte nel capitolo precedente e consente all'utente finale di esportare il Choreography Automaton ricavato dal codice sorgente in formato DOT. È un software \emph{open source} con licenza GPL-3.0\cite{GPL-3.0} scritto completamente in Go, non richiede alcun tipo di setup se non l'installazione iniziale delle dipendenze. È disponibile al download al seguente url: \url{https://github.com/its-hmny/Choreia}\\

\section{Parametri da linea di comando}
Il tool non ha una GUI in quanto, per gli scopi attuali, non è necessaria: infatti non è stato progettato come un tool di uso comune ma come uno strumento per persone interessate e con un minimo di conoscenza pregressa.\\
In ogni caso è possibile tramite \emph{command line} fornire alcuni parametri e flags per un utilizzo \emph{personalizzato}, i parametri disponibili sono:
\begin{table}[h!]
    \centering
    \begin{tabular}{l l l}
        Breve & Esteso   & Descrizione                                               \\
        -i    & --input  & Il \emph{path} del file .go in input                      \\
        -o    & --output & La directory in cui verranno salvati i file i vari automi \\
        -t    & --trace  & Stampa l'AST sullo \texttt{stdout}                        \\
        -h    & --help   & Mostra un messaggio di aiuto con una breve spiegazione    \\
    \end{tabular}
    \caption{La lista di argomenti da linea di comando  accettati da Choreia}
\end{table}

\section{Struttura del progetto}
Il progetto è strutturato su 3 moduli principali, ognuno con uno specifico compito: Il modulo \texttt{fsa} fornisce un'implementazione per gli automi a stati finiti (implementati attraverso un multigrafo), il modulo \texttt{static analysis} gestisce il parsing (fatto tramite la \emph{standard library} golang) e l'estrazione dei metadati dall'AST ed infine il modulo \texttt{transforms} implementa varie operazioni su FSA come per esempio: determinizzazione e minimizzazione (per FSA generici) o la composizione (per automi associati alle local views). \\
La validazione dei dati e l'orchestrazione delle funzionalità fornite dai vari moduli sono gestite nel \texttt{main} che agisce come entry point del programma.\bigskip \\
\begin{forest}
    for tree={
    font=\ttfamily,
    grow'=0,
    child anchor=west,
    parent anchor=south,
    anchor=west,
    calign=first,
    inner xsep=7pt,
    edge path={
            \noexpand\path [draw, \forestoption{edge}]
            (!u.south west) +(7.5pt,0) |- (.child anchor) pic {folder} \forestoption{edge label};
        },
    % style for your file node 
    file/.style={edge path={
                    \noexpand\path [draw, \forestoption{edge}]
                    (!u.south west) +(7.5pt,0) |- (.child anchor) \forestoption{edge label};},
            inner xsep=2pt,font=\small\ttfamily
        },
    before typesetting nodes={
            if n=1
                {insert before={[,phantom]}}
                {}
        },
    fit=band,
    before computing xy={l=15pt},
    }
    [Choreia
        [cmd
                [main.go, file]
        ]
        [go.mod, file]
        [go.sum, file]
        [internal
                [fsa
                        [\text{fsa.go, transition.go}, file]
                ]
                [static analysis
                        [\text{branch.go, channel.go, function.go, loop.go, ...}, file]
                ]
                [transforms
                        [\text{extraction.go, minimization.go, composition.go, ...}, file]
                ]
        ]
        [LICENSE, file]
        [README, file]
    ]
\end{forest}

\section{Flusso d'esecuzione}
Il flusso d'esecuzione riprende le quattro macro fasi definite nella sezione \ref{sec:GoOutline}.\bigskip\\
Come prima cosa il file in input viene validato e parsato, generando un AST (queste funzionalità sono fornite dal modulo \texttt{go/ast} della standard library), l'AST viene poi navigato da un \emph{visitor} che si occupa di raccogliere e organizzare i dati in apposite strutture dati. Durante questa fase vengono generati degli automi a stati finiti non deterministici associati ad ogni funzione dichiarata all'interno del file in input, questi NFA descriveranno l'evoluzione della computazione nello scope di funzione.\bigskip\\
Una volta ottenuti gli automi e i metadati si procede con lo step successivo, ovvero ricavare gli automi associati alle Goroutine presenti nel programma. Per fare questo si utilizza l'algoritmo \ref{alg:Local_Views_Extraction} che attraverso l'\emph{inlining} delle chiamate di funzione e la sostituzione dei parametri formali con quelli attuali ricava in maniera ricorsiva degli automi non deterministici che rappresentano il flusso d'esecuzione completo della Goroutine. Come fase finale di questo step gli automi vengono determinizzati e minimizzati tramite gli algoritmi mostrati nel capitolo \ref{cap:Preliminaries}.\bigskip\\
L'ultimo step dell'esecuzione è quello di generare la global view a partire dalle view locali (e rispettivi automi), esattamente come detto nella sezione \ref{sec:Go_Composition} generiamo l'automa prodotto di tutte le local views e poi eseguiamo l'operazione di sincronizzazione su quest'ultimo, infine esportiamo questo automa \emph{finale} in formato DOT.

\section{Esempi pratici}
Di seguito mostriamo alcuni esempi di programmi Go e il rispettivo Choreography Automaton associatogli da Choreia. % TODO add link to Choreia repo /folders


\subsection{Loop determinato su canale}
In questo esempio possiamo vedere che la Goroutine \texttt{main} inizializza il canale \texttt{channel} ed avvia la Goroutine \texttt{worker}, quest'ultima invia 100 messaggi numerati sul canale condiviso concludendo con la chiusura dello stesso. La chiusura di un canale è importante poichè blocca ogni futuro utilizzo dello stesso da entrambi i lati della comunicazione e permette al costrutto \texttt{for range} di interrompere il loop una volta ricevuti tutti i messaggi in coda. Se \texttt{worker} non chiudesse il canale \texttt{main} si bloccherebbe, alla 101-esima iterazione, aspettandosi di ricevere un messaggio da un canale \emph{abbandonato} e questo creerebbe un problema di \emph{liveness}.
\newpage % TODO avoid forcefully breaking page
\lstinputlisting[language=Go]{Snippets/ForLoop.go}
\bigskip
Di seguito troviamo gli automi delle local views associate rispettivamente alle Goroutine $\texttt{main}_0$ e $\texttt{worker}_1$ ottenuti tramite l'algoritmo di estrazione delle local views. Come si può facilmente osservare rispecchiano perfettamente il flusso d'esecuzione descritto dal codice che in questo caso è particolarmente semplice.\\
Si noti che l'automa della Goroutine $\texttt{worker}_1$ non presenta una transizione del tipo $\rightarrow shared$, come si potrebbe essere indotti a pensare, bensì la label della transizione è $  \rightarrow channel$ poichè l'algoritmo di estrazione gestisce la sostituzione dei parametri formali con quelli attuali.
\begin{figure}[h!]
    \centering
    \begin{tikzpicture}[->,>=stealth',shorten >=1pt,auto,node distance=3cm,scale=1, transform shape, baseline=0]
        \node[state,initial, initial text=$\texttt{main}_0$] (main0) {$0$};
        \node[state, accepting] (main1)[right of=main0] {$1$};
        \node[state, initial, initial text=$\texttt{worker}_1$, accepting] (worker0)[right=9.5cm] {$0$};
        \draw
        (main0) edge node{$\bigtriangleup worker_1$} (main1)
        (main1) edge [loop right] node{$\leftarrow channel$} (main1)
        (worker0) edge [loop right] node{$\rightarrow channel$} (worker0);
    \end{tikzpicture}
\end{figure} \bigskip \\
Il seguente è invece l'automa associato alla global view ottenuto con l'algoritmo di composizione a partire dai due precedenti. Chiaramente in questo caso la sincronizzazione è banale e avviene sulle sole due transizioni di invio e ricezione presenti.
\begin{figure}[h!]
    \centering
    \begin{tikzpicture}[->,>=stealth',shorten >=1pt,auto,node distance=4.5cm,scale=1, transform shape, baseline=0]
        \node[state,initial] (0){$0$};
        \node[state, accepting] (1)[right of=0] {$1$};
        \draw
        (0) edge node{$main_0 \bigtriangleup worker_1$} (1)
        (1) edge [loop right] node{$main_0 \leftarrow worker_1$} (1);
    \end{tikzpicture}
\end{figure}


\subsection{Operazioni condizionali con \texttt{select}}
In questo esempio la Goroutine \texttt{main} crea due canali \texttt{chanA} e \texttt{chanB} e avvia due Goroutine \texttt{responder}, ciascuna con uno dei due canali, dopodichè tramite il costrutto \texttt{select} esegue il primo ramo che presenta un'operazione \emph{non bloccante}, nel caso in cui nessuno dei due canali abbia dei messaggi si mette \emph{in ascolto} aspettando che uno dei due abbia almeno un messaggio in coda. Nel nostro caso le Goroutine \texttt{responder} inviano solo un messaggio e non eseguono altre operazioni tuttavia possiamo assumere che portino avanti computazioni più complesse, inviando alla fine il risultato delle stesse sul canale.\\
\lstinputlisting[language=Go]{Snippets/Select.go}
Gli automi associati alle Goroutine $\texttt{main}_0$, $\texttt{responder}_1$, $\texttt{responder}_2$ sono i seguenti. Si noti come viene \emph{mappato} il costrutto \texttt{select} attraverso gli automi, dal momento che il costrutto permette di \emph{valutare} operazioni su più canali eseguendo solo la prima tra queste che non sia bloccante. Nell'automa a stati finiti questo viene mappato come una scelta non deterministica tra due flussi di esecuzione che poi procedono nei rispettivi \emph{sottografi} ed, eventualmente, si ricongiungono tra loro a fine esecuzione.\bigskip \\
\begin{figure}[h!]
    \centering
    \begin{tikzpicture}[->,>=stealth',shorten >=1pt,auto,node distance=3.5cm,scale=0.9, transform shape, baseline=0]
        \node[state, initial, initial text=$\texttt{main}_0$] (main0) {$0$};
        \node[state] (main1)[right of=main0] {$1$};
        \node[state] (main2)[right of=main1] {$2$};
        \node[state] (main3)[right of=main2, above=0.3cm] {$3$};
        \node[state, accepting] (main4)[right of=main3] {$4$};
        \node[state] (main5)[right of=main2, below=0.3cm] {$5$};
        \node[state, accepting] (main6)[right of=main5] {$6$};

        \node[state, initial, initial where=above, initial text=$\texttt{responder}_1$] (responder10)[below of=main0, right=0.7cm] {$0$};
        \node[state] (responder11)[right of=responder10] {$1$};

        \node[state, initial, initial where=above, initial text=$\texttt{responder}_2$] (responder20)[right of=responder11] {$0$};
        \node[state] (responder21)[right of=responder20] {$1$};
        \draw
        (main0) edge node{$\bigtriangleup responder_1$} (main1)
        (main1) edge node{$\bigtriangleup responder_2$} (main2)
        (main2) edge node[rotate=13, above]{$\leftarrow chanA$} (main3)
        (main3) edge node{$\leftarrow chanB$} (main4)
        (main2) edge node[rotate=-13, below]{$\leftarrow chanB$} (main5)
        (main5) edge node{$\leftarrow chanA$} (main6)

        (responder10) edge node{$\rightarrow chanA$} (responder11)

        (responder20) edge node{$\rightarrow chanB$} (responder21)
        ;
    \end{tikzpicture}
\end{figure}\\
Nel Choreography Automaton generato possiamo vedere come in questo caso l'utilizzo del \emph{prodotto tra automi} permette di considerare tutte le possibili evoluzioni del sistema concorrente che poi verranno elaborate ulteriormente e poste nella forma attuale dalla \emph{sincronizzazione}, che in questo caso genera correttamente entrambi i rami d'esecuzione.\bigskip \\
\begin{figure}[h!]
    \centering
    \begin{tikzpicture}[->,>=stealth',shorten >=1pt,auto,node distance=4.8cm,scale=0.9, transform shape, baseline=0]
        \node[state,initial] (0) {$0$};
        \node[state] (1)[right of=0] {$1$};
        \node[state] (2)[right of=1] {$2$};
        \node[state] (3)[right of=2, above=40pt] {$3$};
        \node[state, accepting] (4)[left of=3, above=40pt] {$4$};
        \node[state] (5)[right of=2, below=40pt] {$5$};
        \node[state, accepting] (6)[left of=5, below=40pt] {$6$};
        \draw
        (0) edge node{$main_0 \bigtriangleup responder_1$} (1)
        (1) edge node{$main_0 \bigtriangleup responder_2$} (2)
        (2) edge node[rotate=22, above]{$main_0 \leftarrow responder_1$} (3)
        (3) edge node[rotate=-22, above]{$main_0 \leftarrow responder_2$} (4)
        (2) edge node[rotate=-22, below]{$main_0 \leftarrow responder_2$} (5)
        (5) edge node[rotate=22, below]{$main_0 \leftarrow responder_1$} (6);
    \end{tikzpicture}
\end{figure}

\subsection{Branching con \texttt{if-then-else}}
In questo esempio la Goroutine \texttt{main} inizializza due canali \texttt{A} e \texttt{B} ed avvia due Goroutine \texttt{dummy} ciascuna con il proprio canale: Una volta completata questa prima fase di setup semplicemente riceve un messaggio da \texttt{A} e uno da \texttt{B}. Si noti che la receive su \texttt{A} è sempre effettuata poichè la condizione dell'\texttt{if} è sempre verificata.
\lstinputlisting[language=Go]{Snippets/If-Then-Else.go}
Gli automi associati alle Goroutine $\texttt{main}_0$, $\texttt{dummy}_1$, $\texttt{dummy}_2$ sono i seguenti. Si noti come viene \emph{mappato} l'\texttt{if-then-else} attraverso gli automi: la biforcazione del ramo then viene mappata correttamente ma allo stesso tempo procede un'esecuzione \emph{lineare} che rappresenta il caso in cui la condizione dell'if non sia verificata. Sempre riguardante la condizione dell'if possiamo banalmente osservare le limitazioni dell'analisi statica descritte al capitolo \ref{cap:Preliminaries}: nonostante la condizione sia sempre verificata Choreia assume del \emph{non determinismo} e biforca il flusso d'esecuzione.
\newpage % TODO avoid forcefully breaking page
\begin{figure}[t!]
    \centering
    \begin{tikzpicture}[->,>=stealth',shorten >=1pt,auto,node distance=3.5cm,scale=0.9, transform shape, baseline=0]
        \node[state,initial, initial text=$\texttt{main}_0$] (main0) {$0$};
        \node[state] (main1)[right of=main0] {$1$};
        \node[state] (main2)[right of=main1] {$2$};
        \node[state] (main3)[above=2cm, right of=main2] {$3$};
        \node[state, accepting] (main4)[below=2cm, right of=main3] {$4$};

        \node[state,initial, initial text=$\texttt{dummy}_1$] (dummy10)[below=2cm] {$0$};
        \node[state, accepting] (dummy11)[right of=dummy10] {$1$};

        \node[state,initial,initial text=$\texttt{dummy}_2$] (dummy20)[right of=dummy11] {$0$};
        \node[state, accepting] (dummy21)[right of=dummy20] {$1$};
        \draw
        (main0) edge node{$\bigtriangleup dummy_1$} (main1)
        (main1) edge node{$\bigtriangleup dummy_2$} (main2)
        (main2) edge node{$\leftarrow B$} (main4)
        (main2) edge node[above, rotate=34]{$\leftarrow A$} (main3)
        (main3) edge node[above, rotate=-32]{$\leftarrow B$} (main4)

        (dummy10) edge node{$\rightarrow A$} (dummy11)

        (dummy20) edge node{$\rightarrow B$} (dummy21);
    \end{tikzpicture}
\end{figure}
Nel Choreography Automaton generato possiamo vedere che la struttura dello stesso è del tutto simile a quella dell'automa associato alla Goroutine $\texttt{main}_0$. Questo è dovuto al fatto che almeno per questi esempi il sistema concorrente è particolarmente semplice e \emph{gerarchico}: il \texttt{main} è la Goroutine dominante mentre le altre Goroutine sono piuttosto banali e immediate, dunque non aggiungono complessità al Choreography Automaton finale. Questi esempi \emph{semplici} sono dovuti alla difficoltà di rappresentazione degli automi e all'incremento nella complessità degli stessi al crescere della complessità del sistema concorrente, tuttavia Choreia è in grado di riconoscere e gestire sistemi concorrenti più complessi ed intricati che sarebbero difficili da riportare graficamente in questa tesi.
\begin{figure}[h!]
    \centering
    \begin{tikzpicture}[->,>=stealth',shorten >=1pt,auto,node distance=4.3cm,scale=0.75, transform shape, baseline=0]
        \node[state,initial] (main0) {$0$};
        \node[state] (main1)[right of=main0] {$1$};
        \node[state] (main2)[right of=main1] {$2$};
        \node[state] (main3)[above=2cm, right of=main2] {$3$};
        \node[state, accepting] (main4)[below=3cm, right=17cm] {$4$};
        \draw
        (main0) edge node{$main_0 \bigtriangleup dummy_1$} (main1)
        (main1) edge node{$main_0 \bigtriangleup dummy_2$} (main2)
        (main2) edge node{$main_0 \rightarrow dummy_2$} (main4)
        (main2) edge node[above, rotate=26]{$main_0 \rightarrow dummy_1$} (main3)
        (main3) edge node[above, rotate=-23]{$main_0 \rightarrow dummy_2$} (main4);
    \end{tikzpicture}
\end{figure}


\subsection{Function call con sostituzione dei parametri}
In questo esempio la Goroutine \texttt{main}, dopo aver creato il canale \texttt{channel} chiama la funzione \texttt{f} passandogli suddetto canale come argomento, questa funzione esegue le seguenti operazioni: invia un messaggio sul canale, avvia la Goroutine \texttt{dummy} condividendo con essa il canale e infine riceve il messaggio inviato precedentemente prima di restituire il controllo a \texttt{main} che conclude ricevendo per la seconda volta dal canale.
\newpage % TODO Avoid forcefully breaking page
\lstinputlisting[language=Go]{Snippets/FunctionCall.go}
Gli automi associati alle Goroutine $\texttt{main}_0$ e $\texttt{dummy}_1$ sono i seguenti. Si noti che la funzione \texttt{f} subisce l'\emph{inlining} durante la fase di estrazione delle local views, difatti possiamo vedere che alcune delle transizioni nel primo automa sono proprio derivanti dall'esecuzione della funzione \texttt{f}.
\begin{figure}[h!]
    \begin{tikzpicture}[->,>=stealth',shorten >=1pt,auto,node distance=3.2cm,scale=1, transform shape, baseline=0]
        \node[state,initial,initial text=$\texttt{main}_0$] (main0) {$0$};
        \node[state] (main1)[right of=main0] {$1$};
        \node[state] (main2)[right of=main1] {$2$};
        \node[state] (main3)[right of=main2] {$3$};
        \node[state, accepting] (main4)[right of=main3] {$4$};

        \node[state,initial,initial text=$\texttt{dummy}_1$] (dummy0)[below of=main1, right=1cm] {$0$};
        \node[state, accepting] (dummy1)[right of=dummy0] {$1$};
        \draw
        (main0) edge node{$\rightarrow channel$} (main1)
        (main1) edge node{$\bigtriangleup dummy$} (main2)
        (main2) edge node{$\leftarrow channel$} (main3)
        (main3) edge node{$\leftarrow channel$} (main4)

        (dummy0) edge node{$\rightarrow channel$} (dummy1);
    \end{tikzpicture}
\end{figure}\\
Choreia, allo stato attuale, non è in grado di generare un automa per il seguente codice, questo è dovuto al fatto che sono presenti diverse criticità all'interno dello stesso: prima di tutto la Goroutine \texttt{main} si sincronizza con se stessa inviando il primo messaggio su \texttt{channel} e ricevendolo poi dallo stesso. Questo, seppur possibile teoricamente in Go, non è \emph{catturabile} dal nostro algoritmo in quanto richiederebbe un adattamento della nozione di prodotto tra automi. Attualmente infatti il prodotto tra automi $A \times B$ non considera il caso con A utilizzato da entrambi i lati come fattore ($A \times A$) e questo comporta una \emph{perdita d'informazione}.\\
Inoltre, anche ammettendo che fosse implementato quanto detto sopra, a questo punto avremmo nel Choreography Automaton ben due sincronizzazione del tipo $main \rightarrow main$ sia nella transizione da 0 a 1, sia nella transizione da 2 a 3 e andrebbe quindi implementato un qualche tipo di algoritmo e/o validazione che eviti di incappare in questi casi particolari.\\
Un'ulteriore problematica legata a questo esempio invece è dovuta al fatto che il nostro algoritmo genererebbe due transizioni da 2 a 3 rispettivamente con label $main \rightarrow main$ e $dummy \rightarrow main$ questo perchè l'algoritmo di composizione attuale non tiene in considerazione che i canali sono gestiti con un politica di \emph{First In First Out (FIFO)} e dunque in questo caso l'unica sincronizzazione possibile tra gli stati 2 e 3 è quella tra main e sè stesso.\bigskip \\
Il Choreography Automaton corretto è il seguente:
\begin{figure}[h!]
    \centering
    \begin{tikzpicture}[->,>=stealth',shorten >=1pt,auto,node distance=4.5cm,scale=0.95,transform shape, baseline=0]
        \node[state,initial] (0) {$0$};
        \node[state] (1)[right of=0] {$1$};
        \node[state] (2)[right of=1] {$2$};
        \node[state, accepting] (3)[right of=2] {$3$};
        \draw
        (0) edge node{$main_0 \bigtriangleup dummy_1$} (1)
        (1) edge node{$main_0 \rightarrow main_0$} (2)
        (2) edge node{$dummy_1 \rightarrow main_0$} (3)
        ;
    \end{tikzpicture}
\end{figure}