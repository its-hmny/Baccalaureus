% "Tool description" chapter
\chapter{Choreia}
Choreia è il tool sviluppato come progetto per questa tesi. Il tool si occupa di tutte le fasi descritte al capitolo descritte precedentemente e consente all'utente finale di esportare il Choreogaphy Automata ricavato dal sorgente in formato DOT.
Choreia è un software libero con licenza GPL-3.0 scritto completamente in Go, non richiede alcun tipo di setup se non l'installazione iniziale delle dipendenze. è disponibile al download al seguente url: \url{https://github.com/its-hmny/Choreia} \\
Il nome \emph{Choreia} deriva dalla medesima parola greca da cui deriva Coreografia parola composta da \emph{choreia}, "danza" e \emph{graphè}, scrittura.

\section{Struttura del progetto}

\section{Parametri da linea di comando}
Il tool non ha una GUI in quanto, almeno per gli scopi attuali, non è necessaria: infatti non è stato progettato come un tool di uso comune ma come uno strumento per persone interessate e con un minimo di conoscenza pregressa.\\
Tuttavia è possibile tramite \emph{command line} fornire alcuni parametri e flags per un utilizzo \emph{personalizzato}, di seguito troviamo una spiegazione di ognuno di essi:
\begin{table}[h!]
    \centering
    \begin{tabular}{l l l}
        Breve & Esteso   & Descrizione                                               \\
        -i    & --input  & Il \emph{path} del file .go in input                      \\
        -o    & --output & La directory in cui verrano salvati i file i vari automi \\
        -t    & --trace  & Stampa l'AST sullo \texttt{stdout}                       \\
        -h    & --help   & Mostra un messaggio di aiuto con una breve spiegazione    \\
    \end{tabular}
    \caption{La lista di argomenti da linea di comando  accettati da Choreia}
\end{table}

\section{Flusso d'esecuzione}

\section{Esempi pratici}