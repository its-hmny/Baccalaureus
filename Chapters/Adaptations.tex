% "Changes and adaptations" chapter
\chapter{Coreografie per Go}
\section{Outline} \label{sec:GoOutline}
L'obiettivo del progetto, ad alto livello, è quello di prendere del codice sorgente \emph{Go} ed estrarre il Choreography Automaton che esprima come le Goroutine interagiscono tra loro durante l'esecuzione del programma, in modo da tale da fornire allo sviluppatore uno strumento per \emph{visualizzare} il sistema concorrente.\\
Concettualmente possiamo dividere questo obiettivo in 4 fasi:
\begin{enumerate}
    \item \textbf{Validazione e parsing}: Il codice sorgente viene validato e trasformato in un \emph{Abstract Syntax Tree} (AST).
    \item \textbf{Estrazione dei metadati}: Viene navigato l'AST estraendo tutte le informazioni necessarie (i metadati relativi a funzioni, canali, ecc) e salvandole in strutture dati appropriate.
    \item \textbf{Derivazione delle local views}: Partendo dai metadati si derivano le local views delle varie Goroutine (gli attori del sistema concorrente).
    \item \textbf{Generazione della coreografia}: Dalle local view ottenute, è necessario generare un singolo Choreography Automata che rappresenti l'intera coreografia del sistema (la view globale).
\end{enumerate}
Questo approccio è chiaramente \emph{bottom-up} mentre l'approccio delle definizioni nel capitolo \ref{cap:Preliminaries} è invece \emph{top-down}: abbiamo visto infatti come sia possibile, partendo dal Choreography Automaton ricavare le singole view locali attraverso l'operazione di \emph{proiezione}.\\
Si rende dunque necessaria l'implementazione di un'operazione opposta alla proiezione che permetta di ottenere una view globale a partire dalle sue singole componenti, ovvero le view locali, questo tipo di operazione riprenderà alcuni concetti dall'operazione di \emph{composizione} opportunamente adattati alla nostra situazione. \\
Il punto cruciale della nostra tecnica è quello di \emph{approssimare} ogni funzione dichiarata all'interno del codice sorgente in un automa a stati finiti che rappresenti il flusso d'esecuzione, in particolare ogni transizione di questo automa rappresenterà un'interazione tra la funzione stessa e il resto della coreografia. \bigskip \\
Per ragioni di chiarezza e semplicità questa tesi si concentra su un sottoinsieme di Go limitato all'utilizzo di canali, iterazione determinata sugli stessi, \texttt{select} statement, canali passati come argomento tra funzioni e creazione di nuove Goroutine. Altre \emph{feature} del linguaggio come ricorsione, \emph{selector expression}, \emph{anonymous functions}, \emph{high order function} o iterazioni su liste e mappe non sono attualmente supportare e potrebbero dunque causare delle inconsistenze nel Choreography Automaton finale.

\subsection{Peculiarità di Go}
Per gli scopi di questa tesi è bene considerare le particolarità di Go in modo da adattare il modello teorico allo stesso e \emph{risolvere} le eventuali incongruenze.\bigskip \\
Mentre aspetti tipici di Go come i canali e il costrutto \texttt{select} non generano particolari conflitti con il modello teorico lo stesso non si può dire per le Goroutine: quest'ultime sono intrinsecamente \emph{gerarchiche}: per ogni programma Go viene avviata sempre e solo una Goroutine (quella che esegue la funzione \texttt{main} e che si comporta come \emph{entry point} del programma stesso) sarà poi questa, durante la sua esecuzione, ad avviarne altre, quest'ultime a loro volta potranno avviarne altre ancora e così via.
Il \emph{conflitto} con il modello teorico deriva dal fatto che nella definizione di Choreography Automata si assume in qualche modo che tutti i partecipanti siano già avviati e pronti a comunicare tra loro mentre per i nostri scopi servirebbe tenere traccia del momento in cui una Goroutine viene avviata in modo da poter definire quando la sua \emph{local view} diventa rilevante nel contesto globale. Senza questo ulteriore controllo si potrebbero verificare delle inconsistenze, per esempio potremmo avere interazioni tra Goroutine prima ancora che queste siano avviate.\bigskip \\
Estendiamo dunque le nozioni di Choreography Automata e di Communicating Finite-State Machine date rispettivamente nelle definizioni \ref{def:Choreography_Automata} e \ref{def:CommunicatingFiniteStateMachine} come segue:

\begin{definition}[Choreography Automata - Estesa] \label{def:Choreography_Automata_Ext}
    Un Choreography Automaton (c-automaton) è un $\epsilon$-free FSA con un insieme di label:\bigskip \\
    \centerline{$\mathcal{L}_{ext} = \mathcal{L}_{int} \cup \{ A \bigtriangleup B | A, B \in \mathcal{P}\}$} \bigskip \\
    dove $A \bigtriangleup B$ indica che $A$ esegue lo \emph{spawn} di $B$ e con $\mathcal{L}_{int}$, $\mathcal{P}$ e $\mathcal{M}$ definiti come nella definizione \ref{def:Choreography_Automata}
\end{definition}

\begin{definition}[Communicating Finite-State Machine - Estesa]
    Una Communicating Finite State Machine (CFSM) è un FSA $C$ con insieme di labels:
    \bigskip \\
    \centerline{$\mathcal{L}_{act} = \{$A B ! m, A B ? m, A $\bigtriangleup$ B $|$ A, B $ \in \mathcal{P},$ m $ \in \mathcal{M}\}$}
\end{definition}

\begin{remark}
    Seppur non interessante per gli scopi di questa tesi è possibile adattare la nozione di proiezione in modo che tenga in considerazione le transizioni del tipo $A \bigtriangleup B$ con $A, B \in \mathcal{P}$
\end{remark}


\section{Estrazione dei metadati}
Una volta ottenuto un AST valido, serve estrarre i metadati necessari da quest'ultimo. Per gli scopi di questa tesi siamo interessati ad estrarre informazioni sui canali dichiarati all'interno del programma, in particolare: il nome della variabile associata, il \emph{tipo} di messaggi che possono essere scambiati sullo stesso e la tipologia del canale (\emph{buffered} o \emph{unbuffered}) con particolare distinzione tra canali dichiarati nello scope globale e quelli dichiarati in uno scope locale. Inoltre siamo interessati ad estrarre metadati dalle \emph{function declarations} come: il nome della funzione, canali dichiarati nello scope della stessa e un FSA che rappresenti il flusso d'esecuzione all'interno della funzione stessa (detto Scope Automaton). Vogliamo anche memorizzare eventuali \emph{parametri formali} della funzione che richiedono trattamenti particolari come canali o \emph{callback functions}, infatti i canali passati come parametro dal chiamante dovranno poi essere \emph{sostituiti} con i parametri formali nel momento in cui la funzione viene chiamata o avviata come Goroutine da un'altra. \bigskip \\
Questa estrazione dei metadati avviene tramite analisi statica, una tecnica descritta nel capitolo \ref{cap:Preliminaries}. Questa tecnica è preferibile rispetto all'analisi dinamica poichè non richiede alcun tipo di esecuzione e quindi protegge da programmi potenzialmente pericolosi e sconosciuti, evita lo scaricamento di eventuali dipendenze per il programma in input e in generale fornisce una visione più ampia e meglio approssimata per i nostri scopi senza richiedere profilazioni multiple. Utilizzando analisi dinamica per questo progetto otterremmo per ogni profilazione eseguita solo un sottografo del Choreography Automaton finale che rappresenta il particolare percorso intrapreso dalla coreografia durante quella esecuzione/profilazione e non l'intera coreografia.

\subsection{Limiti dell'analisi statica} \label{subsec:Static_Analysis_Limits}
Anche per questa fase sorge un'inconsistenza con la definizione formale di Choreography Automaton data in precedenza: l'insieme $\mathcal{M}$ dei messaggi non è determinabile in modo preciso attraverso l'analisi statica. La definizione \ref{def:Choreography_Automata} sembra suggerire che esista un numero finito e definito di messaggi scambiabili tra i vari attori tuttavia questo insieme non è calcolabile con l'approccio utilizzato.
Ricordiamo infatti che questo tipo di analisi viene effettuata utilizzando solo il codice sorgente e ricavando dei dati senza mai eseguire il codice, in generale non è possibile solo attraverso l'analisi statica ricavare il valore esatto di tutte le variabili (e dunque tutti i messaggi inviati), questo perchè tale valore può essere soggetto a svariati \emph{side effect} durante l'esecuzione o può essere legato a parametri temporali (p.e. timestamp), input forniti dall'utente o altri valori ricavabili solo a runtime. Questi \emph{aspetti} non sono \emph{catturabili} attraverso l'analisi statica e dunque devono essere  gestiti in maniera opportuna.\\
\lstinputlisting[language=Go, caption={Codice sorgente per cui non è possibile calcolare $\mathcal{M}$ tramite analisi statica}, label={lst:DynamicAnalysis}]{Snippets/DynamicAnalysis.go}
\bigskip
Come possiamo vedere il programma mostrato è in realtà alquanto banale: la Goroutine \texttt{main} genera un intero random che poi invia su un canale precedentemente condiviso con le due Goroutine \texttt{worker}, uno dei due processi riceverà questo intero lo incrementerà per poi inviarlo nuovamente con un timestamp aggiuntivo. Attraverso l'analisi statica non solo non riusciamo a determinare il valore inviato sul canale \texttt{in} ma nemmeno entrambi i valori inviati sul canale \texttt{out}. \bigskip \\
La soluzione da noi adottata è in realtà molto semplice ma permette di mantenere una sufficiente espressività del modello: generalizziamo $\mathcal{M}$ all'\emph{insieme dei tipi dei messaggi scambiati}, i tipi infatti possono essere inferiti e ricavati senza particolari problemi per mezzo di analisi statica. Nel caso della figura sopra $\mathcal{M}$ sarà definito come: $\mathcal{M} = \{ int, payload \}$ e le label nel Choreography Automaton associato saranno del tipo $worker \leftarrow main : int$ oppure $main \leftarrow worker : payload$.

\section{Derivazione delle local views}
Una volta ottenuti tutti i metadati necessari sorge la necessità di derivare dagli stessi le local views di ogni Goroutine creata durante l'esecuzione del programma, da queste poi potremmo procedere alla costruzione della global view.\bigskip\\
Partiamo definendo la struttura dell'automa che verrà prodotto durante queste fase, rispetto alla fase precedente dove l'automa è associato ad una singola funzione qui l'automa ottenuto sarà associato ad un intera Goroutine. Per i nostri scopi non siamo interessati all'intero flusso d'esecuzione bensì solo alle interazioni con i canali, allo \emph{spawn} di nuove Goroutine e alle \emph{function call} effettuate. \\
\begin{definition}[Goroutine FSA]
    Un Goroutine FSA è un fsa G con insieme di labels:
    \begin{center}
        $\mathcal{L} = \{ \leftarrow c, \rightarrow c, \bigtriangleup f | \text{c è un canale ed f è una funzione} \}$
    \end{center}
    le label indicano rispettivamente la ricezione da un canale, l'invio su un canale e lo spawn di una nuova Goroutine con entrypoint f
\end{definition}
Un pattern tipico di Go è quello di avviare Goroutine passando alle stesse il canale/i su cui comunicare, questo comporta un meccanismo di sostituzione dei \emph{parametri formali} con i \emph{parametri attuali}. L'operazione è in realtà molto semplice: durante la fase di estrazione dati salviamo gli argomenti di tipo \texttt{chan} previsti dalla \emph{function declaration}. Su questi parametri formali possono essere svolte delle operazioni di send e receive, le quali compariranno all'interno dell'automa associato alla funzione stessa. Nel momento in cui questa funzione viene chiamata o avviata come Goroutine andremo a sostituire tutte le label del tipo $\rightarrow \texttt{formal}_0$ o $\leftarrow \texttt{formal}_0$ con $\rightarrow \texttt{actual}_0$ o $\leftarrow \texttt{actual}_0$, ripetendo questa operazione per tutti i parametri formali rilevati. \\
Un altro aspetto da tenere in considerazione riguarda le chiamate di funzione effettuate, in questo caso per avere una migliore approssimazione serve che l'automa associato al \emph{chiamato} venga \emph{fuso} all'automa del chiamante. Questo tipo di operazione è simile all'\emph{inlining}: un'ottimizzazione usata dai compilatori per evitare l'overhead della \emph{function call} quando possibile. Esattamente come nell'inlining sostituiamo la transizione con la chiamata di funzione con l'automa associato alla funzione chiamata.\\
In particolare: rimuoviamo la transizione $p \xrightarrow{f} q$ nell'automa del chiamante, copiamo in quest'ultimo tutti gli stati e transizioni relative all'automa del chiamato. Una volta fatto ciò, aggiungiamo le seguenti transizioni: $p \xrightarrow{\epsilon} s_{f0}$ per collegare il chiamante allo stato iniziale del chiamato e $\forall s_{fx} \in \mathcal{F}_f$ aggiungiamo $s_{fx} \xrightarrow{\epsilon} q$ per \emph{mappare} il ritorno del controllo al chiamante. \bigskip \\
Vediamo ora l'algoritmo per l'estrazione delle local views. In questo caso l'esistenza di una gerarchia \emph{intrinseca} nelle Goroutine ci è di aiuto poichè ci permette di assumere un input iniziale per l'algoritmo, ovvero la Goroutine che esegue la funzione \texttt{main}.
\begin{algorithm}
    \caption{Derivazione delle local views} \label{alg:Local_Views_Extraction}
    \begin{algorithmic}
        \State $grSet \gets \{ main \}$
        \While{$\exists$ gr $\in grSet$ non marcato} \Comment{Per ogni Goroutine trovata}
        \ForEach {$t \in gr$} \Comment{Per ogni transizione nell'FSA di questa Goroutine}
        \If {$t.kind = Call$}
        \State espande i parametri formali con quelli attuali
        \State inline del chiamato all'interno del chiamante
        \ElsIf {$t.kind = Spawn$}
        \State espande i parametri formali con quelli attuali
        \State $grSet \gets grSet \cup \{ t.target \}$
        \EndIf
        \EndFor
        \State marca $gr$
        \EndWhile
    \end{algorithmic}
\end{algorithm}\\
Applicando l'algoritmo di sopra al codice sorgente del listing \ref{lst:DynamicAnalysis} otteniamo i seguenti automi rispettivamente per le Goroutine $main_0$, $worker_1$ e $worker_2$:
\begin{figure}[ht]
    \centering
    \begin{tikzpicture}[->,>=stealth',shorten >=1pt,auto,node distance=3.5cm,scale=1, transform shape, baseline=0]
        \node[state,initial,initial text=$\texttt{main}_0$] (A0) {$0$};
        \node[state] (A1) [right of=A0] {$1$};
        \node[state,accepting] (A2) [right of=A1] {$2$};
        \node[state] (A3) [right of=A2] {$3$};
        \node[state,initial,,initial text=$\texttt{worker}_1$,accepting] (B0) [below of=A0] {$0$};
        \node[state] (B1) [right of=B0] {$1$};
        \node[state,initial,,initial text=$\texttt{worker}_2$,accepting] (C0) [right of=B1] {$0$};
        \node[state] (C1) [right of=C0] {$1$};
        \draw
        (A0) edge node{$\bigtriangleup$ worker} (A1)
        (A1) edge node{$\bigtriangleup$ worker} (A2)
        (A2) edge [bend left]node{$\rightarrow$ in} (A3)
        (A3) edge [bend left]node{$\leftarrow$ out} (A2)
        (B0) edge [bend left]node{$\leftarrow$ in} (B1)
        (B1) edge [bend left]node{$\rightarrow$ out} (B0)
        (C0) edge [bend left]node{$\leftarrow$ in} (C1)
        (C1) edge [bend left]node{$\rightarrow$ out} (C0);
    \end{tikzpicture}
    \caption{La local views per le Goroutine \texttt{main} e \texttt{worker} del listing \ref{lst:DynamicAnalysis}}
    \label{fig:Extracted_Local_View}
\end{figure}
\newpage % TODO avoid forcecully breaking pages

\section{Generazione della coreografia} \label{sec:Go_Composition}
Ottenuti gli automi associati alle singole Goroutine, che ricordiamo essere nel nostro caso le local views, ci si presenta il problema di capire come \emph{fondere} questi in un unico automa che rappresenti la coreografia e rispetti la definizione \ref{def:Choreography_Automata_Ext}.\bigskip \\
Il problema che si presenta in questa fase è il seguente: in Go un canale può essere condiviso tra \emph{n} Goroutine ed ognuna di queste può inviare e ricevere a suo piacimento sullo stesso. Prendendo per esempio la figura precedente i dati inviati dalla Goroutine $main_0$ sul canale \texttt{in} possono essere ricevuti da $worker_0$ o da $worker_1$ in maniera non deterministica e pertanto non è possibile determinare il destinatario effettivo di un messaggio. Inoltre la complessità del problema peggiora se generalizziamo ad un caso con \emph{n} sender ed \emph{m} receiver o, peggio ancora, con \emph{n} attori che eseguono sia operazione di send che di receive sullo stesso canale condiviso (questo caso apre anche alla possibilità che una Goroutine riceva un messaggio inviato precedentemente da lei stessa). \bigskip \\
La nostra soluzione considera i canali come delle \emph{interfacce}, quindi tutte le transizioni con operazioni su canali all'interno di un Goroutine FSA sono considerate come interazioni tra un attore (la Goroutine) e un'interfaccia (il canale). Nel capitolo \ref{cap:Preliminaries} però abbiamo già definito l'operazione di \emph{composizione} che permette di unire molteplici Choreography Automata, comunicanti tra loro mediante interfacce, in uno unico.\\
Adattiamo dunque la nozione di composizione ai Goroutine FSA, anzichè ai Choreography Automata: l'operazione di prodotto tra automi rimane invariata rispetto a quanto detto nella definizione \ref{def:FSA_Product} mentre definiamo ora l'operazione di sincronizzazione come segue. Si noti che indichiamo con la notazione $p_A \xrightarrow{\rightarrow c} q_A$ che esiste una transizione dallo stato p allo stato q con label $\rightarrow c$ nell'automa associato alla Goroutine A.
\begin{equation*}
    \mathcal{S}(A \times B) =
    \begin{cases}
        p \xrightarrow{A \rightarrow B: m} q \hfill \text{ se } \exists p_A \xrightarrow{\rightarrow c} q_A,  p_B \xrightarrow{\leftarrow c} q_B \\
        p \xrightarrow{A \bigtriangleup B} q \hfill \text{ se } \exists p_A \xrightarrow{\bigtriangleup B}q_A                                    \\
        \text{nessuna transizione} \hfill \text{altrimenti}
    \end{cases}
\end{equation*}
In questo modo l'automa prodotto ci permette di considerare tutti le \emph{possibili evoluzioni} della computazione concorrente mentre l'operazione di sincronizzazione ci permette di \emph{filtrare} e \emph{unire} solo le evoluzioni che portano ad una interazione tra due Goroutine sullo stesso canale. Queste tipo di interazioni, assieme a quelle di spawn, saranno le sole transizioni ammesse nell'automa finale dal momento che tutte le transizioni che non rispettano i casi definiti sopra vengono eliminate. Così facendo otteniamo in output dall'operazione di composizione un automa che rispetta la definizione \ref{def:Choreography_Automata_Ext}.\bigskip \\
Si noti che se una Goroutine non interagisce con il resto della coreografia, per esempio comunicando su un canale non utilizzato dagli altri attori, allora essa non comparirà all'interno della coreografia finale. Questa è una conseguenza derivante dal fatto che l'operazione di composizione \emph{filtra} le possibili evoluzioni della composizione mantenendo solo quelle che presentano l'interazione/i tra due Goroutine e dunque una Goroutine che non interagisce con il resto della coreografia viene \emph{scartata} durante questa \emph{trasformazione}.\bigskip \\
Concludiamo questo capitolo con il Choreography Automaton ottenuto a partire dalle local views mostrate in figura \ref{fig:Extracted_Local_View} mediante l'algoritmo di composizione descritto in questa sezione.
\begin{figure}[h!]
    \centering
    \begin{tikzpicture}[->,>=stealth',node distance=4.2cm,scale=1, transform shape,]
        \node[state, initial] (0) {$0$};
        \node[state] (1) [right of=0] {$1$};
        \node[state, accepting] (2) [right of=1] {$2$};
        \node[state] (3) [right of=2] {$3$};

        \draw
        (0) edge node[above]{$main \bigtriangleup worker_1$} (1)
        (1) edge node[above]{$main \bigtriangleup worker_2$} (2)
        (2) edge[bend left] node[above,align=center]{$main \rightarrow worker_1 : int$\\$main \rightarrow worker_2 : int$} (3)
        (3) edge[bend left] node[below,align=center]{$worker_1 \rightarrow main : payload$\\$worker_2 \rightarrow main : payload$} (2);
    \end{tikzpicture}
    \caption{Il Choreography Automaton associato al codice sorgente del listing \ref{lst:DynamicAnalysis}}
\end{figure}