% "Technology stack" chapter
\chapter{Tecnologie e librereie utilizzate}
\section{Go (golang)}
Go (anche chiamato {golang}) e' un linguaggio di programmazione \emph{general purpose} open source sviluppato nel 2007 da Robert Griesemer, Rob Pike e Ken Thompson e poi supportato da Google stesso negli anni a seguire. Fortemente ispirato al C presenta una sintasssi minimale e molto semplice, Go e' \emph{statically typed} e fornisce un \emph{Garbage Collector} lasciando comunque all'utente la possibilita' di interagire con i puntatori e allocare dinamicamente la memoria in modo autonomo.\\
Alcuni dei problemi che Go mira a risolvere sono
\begin{itemize}
    \item \textbf{Controllo restrittivo delle dipendenze}: Infatti per evitare di appensantire l'eseguibile finale Go rifiuta di compilare moduli o file dove non tutte le dipendenze importate vengono utilizzate
    \item \textbf{Compilazione piu veloce}: Grazie a quanto detto sopra e alla sintassi estremamente semplice e snella il compilatore riesce a diminuire drasticamente il tempo richiesto alla compilazione mantenendo tutti i vantaggi dell'avere le eventuali ottimizzazioni della compilazione
    \item \textbf{Approccio semplificato alla concorrenza}: Il linguaggio utilizza le Goroutine, dei \emph{processi leggeri}, le quali permettono un utilizzo semplificato e piu' accessibile della programmazione concorrente
\end{itemize}
Altre feauture del linguaggio degne di nota sono: il package manager della \emph{standard library} e l'ecosistema di pacchetti totalmente ditribuito e librerie disponibili, e la grande varieta' di architetture supportate (comprensive di \emph{microcontroller} e \emph{embedded systems}).\\
Go e' stato utilizzato nello sviluppo di tecnologie celeberrime come Docker e Kubernetes e attualmente viene utilizzato da grandi aziende quali Google, MongoDB, Dropbox, Netflix, Uber e altri.

%TODO ADD CITATION TO CORINNE PAPER
\section{Graphviz e DOT}
Vista la necessita' di \emph{rappresentare} in qualche modo il Choreography Automata finale e gli eventuali risultati intermedi si e' reso necessario l'utilizzo di un qualche tipo di \emph{meccanismo di serializzazione}. Fortunatamente considerando la somiglianza tra Finite State Automata e Grafi, si puo' dire che i secondi siano una generalizzazione dei primi, abbiamo potuto riutilizzare tool e strumenti pensati \emph{principalmente} per quest'ultimi.\\
Abbiamo quindi scelto di usare Graphviz, una librearia open source per la visualizzazione di grafi il quale utiliza DOT, un formato specificatamente progettato per la descrizione dei grafi.\\
La scelta e' ricaduta su DOT e Graphviz per alcuni motivi principali:
\begin{itemize}
    \item Il linguaggio DOT e' \emph{human readable} e particolarmente facile da comprendere, inoltre Graphviz permette di \emph{convertire} o \emph{esportare} in formati di uso piu comune come PNG o SVG
    \item Permette un utilizzo combinato con \emph{Corinne},un tool grafico per la visualizzazione e manipolazione dei Choreography Automata
    \item Essendo Graphviz ormai uno standard \emph{de facto} sono presenti librerie e binding che ne permettono l'utilizzo con moltissimi linguaggi di programmazione, tra cui Go
\end{itemize}
Di seguto un mostriamo un esempio banale di Choreography Automata definito attraverso il linguaggio DOT.
\begin{lstlisting}[caption=Rappresentazione in DOT dell'automa in figura \ref{fig:ChoreographyAutomata_Example}]
    digraph DOT_Graph_Example {
        node [shape=circle, fontsize=20]
        edge [length=100, fontcolor=black]
      
        q0 -> q1[label="A->B:tic"];
        q1 -> q2[label="B->C:count"];
        q2 -> q0[label="C->A:toc"];
    }
\end{lstlisting}
Chiaramente i Choreography Automata generati da Choreia non saranno cosi' semplici e immediati, ciononostante dovrebbe essere comunque possibile interagirvi e comprenderli.

\section{Altre librerie} % TODO Keep this section 