% "Technology stack" chapter
\chapter{Tecnologie e librereie utilizzate}
\section{Go (golang)}
Go (anche chiamato {golang}) e' un linguaggio di programmazione \emph{general purpose} open source sviluppato nel 2007 da Robert Griesemer, Rob Pike e Ken Thompson e poi supportato da Google stesso negli anni a seguire. Fortemente ispirato al C presenta una sintasssi minimale e molto semplice, Go e' \emph{statically typed} e fornisce un \emph{Garbage Collector} lasciando comunque all'utente la possibilita' di interagire con i puntatori e allocare dinamicamente la memoria in modo autonomo.\\
Alcuni dei problemi che Go mira a risolvere sono
\begin{itemize}
    \item \textbf{Controllo restrittivo delle dipendenze}: Infatti per evitare di appensantire l'eseguibile finale Go rifiuta di compilare moduli o file dove non tutte le dipendenze importate vengono utilizzate
    \item \textbf{Compilazione piu veloce}: Grazie a quanto detto sopra e alla sintassi estremamente semplice e snella il compilatore riesce a diminuire drasticamente il tempo richiesto alla compilazione mantenendo tutti i vantaggi dell'avere le eventuali ottimizzazioni della compilazione
    \item \textbf{Approccio semplificato alla concorrenza}: Il linguaggio utilizza le Goroutine, dei \emph{processi leggeri}, le quali permettono un utilizzo semplificato e piu' accessibile della programmazione concorrente
\end{itemize}
Altre feauture del linguaggio degne di nota sono: il package manager della \emph{standard library} e l'ecosistema di pacchetti totalmente ditribuito e librerie disponibili, e la grande varieta' di architetture supportate (comprensive di \emph{microcontroller} e \emph{embedded systems}).\\
Go e' stato utilizzato nello sviluppo di tecnologie celeberrime come Docker e Kubernetes e attualmente viene utilizzato da grandi aziende quali Google, MongoDB, Dropbox, Netflix, Uber e altri.

\section{Graphviz e Linguaggio DOT}
TODO