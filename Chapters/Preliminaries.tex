% Preliminaries and Notation chapter
\chapter{Nozioni preliminari e notazione} \label{cap:Preliminaries}
% Finite State Automata (+ related) section
\section{FSA non deterministici e deterministici}
Prima di introdurre le coreografie e i Choreography Automata è necessario fare un breve richiamo di alcune nozioni fondamentali quali la nozione di Automa a Stati Finiti (FSA)\cite{Linguaggi_di_Prorgammazione} e alcune operazioni possibili sugli stessi.
Gli automi a stati finiti sono la descrizione di un sistema dinamico che evolve nel tempo, esiste un parallelo tra gli automi e i calcolatori moderni, per esempio il flusso d'esecuzione di un programma può essere rappresentato attraverso un automa.
Alcune applicazioni pratiche di questi automi possono essere, per esempio, regular expression (RegEx o RegExp), lexer e parser ma possono essere impiegati, come vedremo in questa tesi, anche nel campo dei sistemi concorrenti.\\
Si noti che sebbene per gli scopi di questa tesi gli automi a stati finiti siano dei costrutti sufficientemente potenti esistono tuttavia altre classi di automi, espressivamente più potenti, ai quali corrispondono altrettante classi di linguaggi (si veda, per esempio, gli automi a pila). Tuttavia gli automi a stati finiti sono tra i più semplici e immediati.

\begin{definition}[Finite State Automata]
    Un automa a stati finiti (FSA) è una tupla A = $\langle \mathcal{S}, s_0, \mathcal{F}, \mathcal{L}, \delta \rangle$ dove:
    \begin{itemize}
        \item $\mathcal{S}$ è un insieme finito di stati
        \item $s_0 \in \mathcal{S}$ è lo stato iniziale dell'automa
        \item $\mathcal{F}$ è l'insieme degli stati finali o di accettazione ($\mathcal{F} \subseteq \mathcal{S}$)
        \item $\mathcal{L}$ è l'alfabeto finito, talvolta detto anche insieme di label ($\epsilon \notin \mathcal{L}$)
        \item $\delta : \mathcal{S} \times (L \cup \{\epsilon\}) \rightarrow \mathcal{P}(\mathcal{S})$ è la funzione di transizione ($\epsilon$ denota la stringa vuota)
    \end{itemize}
\end{definition}

\begin{remark}
    È possibile trovare una definizione alternativa in cui non è presente l'insieme degli stati di terminazione (o di accettazione) $\mathcal{F}$. In tal caso assumiamo che ogni $s \in \mathcal{S}$ sia uno stato di accettazione.
\end{remark}

\begin{remark}
    Va notato anche che questa definizione coincide con quella di automa a stati finiti \emph{non deterministico}, solitamente indicato in letteratura con la sigla NFA (\emph{Non Deterministic Finite Automata}). Una sottoclasse particolarmente rilevante è quella dei DFA (\emph{Deterministic Finite Automata}) che andremo a definire nelle prossime pagine.
\end{remark}
Il seguente è un NFA N un grado di riconoscere la Regular Expression $(a|b)^*ba$. Si noti che questo è solo un \emph{possibile} NFA in grado di riconoscere il linguaggio dato ma ne esistono infiniti altri equivalenti ad esso.
\begin{figure}[h!]
    \centering
    \begin{tikzpicture}[->,>=stealth',node distance=2cm,scale=1, transform shape]
        \node[state,initial] (q0) {$q_0$};
        \node[state] (q1) [right of=q0] {$q_1$};
        \node[state] (q2) [above right of=q1] {$q_2$};
        \node[state] (q3) [right of=q2] {$q_3$};
        \node[state] (q4) [below right of=q1] {$q_4$};
        \node[state] (q5) [right of=q4] {$q_5$};
        \node[state] (q6) [above right of=q5] {$q_6$};
        \node[state] (q7) [right of=q6] {$q_7$};
        \node[state] (q8) [right of=q7] {$q_8$};
        \node[state, accepting] (q9) [right of=q8] {$q_9$};
        \draw
        (q0) edge[above] node{$\epsilon$} (q1)
        (q0) edge[bend left=60, above] node{$\epsilon$} (q7)
        (q1) edge[above] node[rotate=45]{$\epsilon$} (q2)
        (q2) edge[above] node{$a$} (q3)
        (q3) edge[above] node[rotate=-45]{$\epsilon$} (q6)
        (q1) edge[above] node[rotate=-45]{$\epsilon$} (q4)
        (q4) edge[above] node{$b$} (q5)
        (q5) edge[above] node[rotate=45]{$\epsilon$} (q6)
        (q6) edge[above] node{$\epsilon$} (q7)
        (q6) edge[above] node{$\epsilon$} (q1)
        (q7) edge[above] node{$b$} (q8)
        (q8) edge[above] node{$a$} (q9);
    \end{tikzpicture}
    \caption{Un possibile NFA che riconosce la RegEx $(a|b)^*ba$}
    \label{fig:NonDeterministic_Example}
\end{figure}

\begin{definition}[Deterministic Finite Automata]
    Un automa a stati finiti deterministico è una tupla D = $\langle \mathcal{S}, s_0, \mathcal{F}, \mathcal{L}, \delta \rangle$ dove $\delta : \mathcal{S} \times \mathcal{L} \rightarrow \mathcal{S}$
\end{definition}
Le varianti deterministiche si distinguono dalle loro controparti non deterministiche dal fatto che non ammettono né l'utilizzo di $\epsilon$ transizioni, né l'utilizzo di transizioni \emph{uscenti}, dallo stesso stato, con la medesima etichetta. Sebbene queste due varianti siano tra loro equivalenti, l'utilizzo di una variante rispetto all'altra può essere determinato da fattori come: necessità di una maggiore elasticità (gli NFA sono meno stringenti rispetto ai DFA) o di una migliore chiarezza (i DFA sono più immediati e semplici).\\
In ogni caso è sempre possibile, dato un NFA qualunque, ottenere un DFA ad esso equivalente anche se quest'ultimo spesso ha un numero maggiore di stati rispetto all' NFA di partenza. L'algoritmo che permette di fare questa trasformazione fa uso estensivo di \emph{$\epsilon$ closure}\cite{Linguaggi_di_Prorgammazione} e della funzione \emph{mossa}\cite{Linguaggi_di_Prorgammazione} che andremo a definire di seguito:

\begin{definition}[$\epsilon$ closure]
    Fissato un NFA N = $\langle \mathcal{S}, s_0, \mathcal{F}, \mathcal{L}, \delta \rangle$ ed uno stato $s \in \mathcal{S}$ si dice $\epsilon$ closure di s, indicata con $\epsilon$-clos(s), il più piccolo $\mathcal{R} \subseteq \mathcal{S}$ tale che:
    \begin{itemize}
        \item $s \in \epsilon$-clos(s)
        \item se x $\in \epsilon$-clos(s) allora $\delta(x, \epsilon) \subseteq \epsilon$-clos(s)
    \end{itemize}
\end{definition}

\begin{remark}
    Se $\mathcal{X}$ è un insieme di stati definiamo $\epsilon$-clos($\mathcal{X}$) come $\bigcup_{x \in \mathcal{X}} \epsilon \mbox{-clos(x)}$.
\end{remark}

\begin{definition}[Mossa]
    Dato un insieme di stati $\mathcal{X} \subseteq \mathcal{S}$ e un simbolo $\alpha \in \mathcal{L}$ definiamo la funzione \emph{mossa}: $\mathcal{P(S)} \times \mathcal{L} \longrightarrow \mathcal{P(S)}$ tale che: \emph{mossa}$(\mathcal{X}, \alpha) = \bigcup_{x \in \mathcal{X}}\delta(x, \alpha)$, ovvero l'insieme di stati raggiungibili da un dato insieme di stati di partenza, leggendo in input $\alpha$.
\end{definition}
L'algoritmo che permette di ricavare un DFA da un qualsiasi NFA è il seguente:
\begin{algorithm}
    \caption{Costruzione per sottoinsiemi} \label{alg:SubsetConstruction_Algorithm}
    \begin{algorithmic}
        \State $x \gets$ $\epsilon$-clos($s_0$) \Comment{Lo stato iniziale del DFA}
        \State $\mathcal{T} \gets \{ x \}$                   \Comment{Un insieme di $\epsilon$-clos}
        \While{$\exists$ t $\in \mathcal{T}$ non marcato}
        \State marca(t)
        \ForEach {$\alpha \in \mathcal{L} $}
        \State $r \gets \epsilon$-clos(mossa(t, $\alpha$))
        \If {$r \notin \mathcal{T}$}
        \State $\mathcal{T} \gets \mathcal{T} \cup \{r\}$
        \EndIf
        \State $\delta(t, \alpha) = r$  \Comment{Denota che $\delta$ del DFA sarà definita su t ed $\alpha$ con output r}
        \EndFor
        \EndWhile
    \end{algorithmic}
\end{algorithm}\\
Si noti che $x$, $\mathcal{T}$ e $\delta$ saranno rispettivamente lo stato iniziale, l'insieme degli stati e la funzione di transizione del DFA corrispondente, $\mathcal{F}$ sarà invece l'insieme di tutti i $t \in \mathcal{T}$ che al loro interno contengono almeno uno stato finale dell'NFA di partenza mentre $\mathcal{L}$ rimane invariato. Quindi il DFA ottenuto in output sarà D = $\langle \mathcal{T}, x, \mathcal{F}, \mathcal{L}, \delta \rangle$.\bigskip \\
L'automa presentato di seguito riconosce sempre la RegEx $(a|b)^*ba$ ed è del tutto equivalente a quello in figura \ref{fig:NonDeterministic_Example} ma è stato ottenuto proprio da quest'ultimo tramite l'algoritmo di \emph{Costruzione per sottoinsiemi}.
\newpage
\begin{figure}[t!]
    \centering
    \begin{tikzpicture}[->,>=stealth',shorten >=1pt,auto,node distance=3cm,scale=1, transform shape, baseline=0]
        \node[state,initial] (A) {$A$};
        \node[state] (B) [above right of=A] {$B$};
        \node[state] (C) [right of=A] {$C$};
        \node[state, accepting] (D) [above right of=C] {$D$};
        \draw
        (A) edge[above] node{$a$} (B)
        (A) edge[above] node{$b$} (C)
        (B) edge[loop above] node{$a$} (B)
        (B) edge[above] node{$b$} (C)
        (C) edge[loop below] node{$b$} (C)
        (C) edge[bend right, above] node{$a$} (D)
        (D) edge[above] node{$a$} (B)
        (D) edge[bend right,above] node{$b$} (C);
    \end{tikzpicture}
    \parbox{5cm}{
        \begin{tabular}{c c}
            Stati NFA           & Stati DFA \\
            \{ 0,1,2,4,7 \}     & A         \\
            \{ 1,2,3,4,6,7 \}   & B         \\
            \{ 1,2,4,5,6,7,8 \} & C         \\
            \{ 1,2,3,4,6,7,9 \} & D
        \end{tabular}
    }
    \caption{Il DFA equivalente all'NFA in figura \ref{fig:NonDeterministic_Example}}
    \label{fig:Deterministic_Example}
\end{figure}

\subsection{Minimizzazione}
Nell'ambito della teoria degli automi esistono una serie di operazioni e trasformazioni che è possibile effettuare, per esempio la composizione di più automi, tuttavia nel nostro caso poniamo particolare riguardo alla minimizzazione. Capita spesso infatti che un automa abbia un numero di stati maggiore del necessario e che alcuni di questi stati siano equivalenti tra loro (e dunque duplicati). Attraverso la minimizzazione è possibile \emph{fondere} insieme questi stati tra loro ottenendo un automa più snello (in numero di stati e transizioni) e più facile da comprendere. Si noti che questo problema degli stati duplicati non sorge solo dalla progettazione umana ma può anche essere un \emph{side effect} di algoritmi come quello di costruzione per sottoinsiemi mostrato sopra.\\
L'algoritmo più conosciuto per minimizzare un automa è detto \emph{Algoritmo di Riempimento a Scala}\cite{Linguaggi_di_Prorgammazione} e, di seguito, vedremo il suo funzionamento. Occorre fare un'importante premessa prima di introdurre l'algoritmo, il funzionamento dello stesso è legato al fatto che la funzione di transizione $\delta$ sia definita su ogni $\alpha \in \mathcal{L}$, la letteratura distingue gli automi \emph{incompleti}, che non verificano questa condizione, da quelli \emph{completi}.\\
Negli automi incompleti la funzione di transizione è parziale e dunque sorgono dei problemi nel momento in cui cerchiamo di minimizzarli, una soluzione molto semplice è quella di usare uno \emph{stato di errore} $\mathcal{E}$ (detto anche \emph{stato pozzo}). Essenzialmente si va a completare la funzione di transizione nei casi mancanti (non definiti) con una transizione verso questo stato di errore, allo stesso tempo tutte le transizioni uscenti da questo stato di errore tornano sullo stesso ($\forall_{ \alpha \in \mathcal{L}} \delta(\mathcal{E}, \alpha) = \mathcal{E}$). Il nome di stato pozzo deriva infatti dal fatto che una volta raggiunto non è possibile uscirne.\\
L'intuizione alla base dell'algoritmo di riempimento a scala è la seguente, valutiamo le singole coppie $(p, q)$ con $p, q \in \mathcal{S}$ e cerchiamo un $\alpha \in \mathcal{L}$ tale che lo stato p si comporti diversamente rispetto allo stato q, questo ci permette di dimostrare che p e q non sono equivalenti e dunque non hanno ragione di essere fusi insieme. Alla fine dell'esecuzione tutte le coppie di stati che non saranno distinte tra loro indicheranno degli stati equivalenti. L'algoritmo di Riempimento della Tabella a Scala è definito come segue:
\begin{algorithm}
    \caption{Riempimento della Tabella a Scala} \label{alg:Minimization_Algorithm}
    \begin{algorithmic}
        \State Inizializza la tabella a scala con le coppie (p,q)
        \State Marca con marca $x_0$ le coppie $(m,n)$ con $m \in \mathcal{F}$ e $n \notin \mathcal{F}$
        \While{$\exists$ almeno un marchio $x_i$ all'iterazione i}
        \If{$\exists \alpha \in \mathcal{L}, \exists p, q \in \mathcal{S}$ tale che $(\delta(p, \alpha), \delta(q, \alpha))$ è marcata} 
        \State Marca $(p, q)$ con marca $x_i$
        \EndIf
        \State Considera all'iterazione seguente solo gli stati non marcati
        \EndWhile
    \end{algorithmic}
\end{algorithm}\\
Di seguito troviamo la versione \emph{minimizzata} dell'automa in figura \ref{fig:Deterministic_Example} con l'algoritmo di \emph{Riempimento della Tabella a Scala}. Anche questo automa riconosce sempre la RegEx $(a|b)^*ba$.
\begin{figure}[ht]
    \centering
    \begin{tikzpicture}[->,>=stealth',shorten >=1pt,auto,node distance=3cm,scale=1, transform shape, baseline=0]
        \node[state,initial] (AB) {$A,B$};
        \node[state] (C) [right of=A] {$C$};
        \node[state, accepting] (D) [right of=C] {$D$};
        \draw
        (AB) edge[loop above] node{$a$} (AB)
        (AB) edge[above] node{$b$} (C)
        (C) edge[loop above] node{$b$} (C)
        (C) edge[bend left, above] node{$a$} (D)
        (D) edge[bend left, above] node{$b$} (C)
        (D) edge[bend left, below] node{$b$} (A);
    \end{tikzpicture}
    \space \space \space
    \parbox{5cm}{
        \begin{tabular}{ | c | c | c | c |}
            \hline B &       & $//$  & $//$  \\
            \hline C & $x_1$ & $x_1$ & $//$  \\
            \hline D & $x_0$ & $x_0$ & $x_0$ \\
            \hline   & A     & B     & C     \\
            \hline
        \end{tabular}
    }
    \caption{Il DFA minimizzato ottenuto da quello in figura \ref{fig:Deterministic_Example}}
    \label{fig:Minimization_Example}
\end{figure}\\

\subsection{Prodotto}
L'ultima operazione su automi a stati finiti che introduciamo è il \emph{prodotto} tra automi. Solitamente questa operazione viene usata per ricavare, a partire da due o più linguaggi e rispettivi automi, un automa che riconosca l'unione e/o l'intersezione di tali linguaggi.
\begin{definition}[Prodotto di automi] \label{def:FSA_Product}
    Siano $A_1 = \langle \mathcal{S}_1, s_{01}, \mathcal{F}_1, \mathcal{L}_1, \delta_1 \rangle$ e \\ $A_2 = \langle \mathcal{S}_2, s_{02}, \mathcal{F}_2, \mathcal{L}_2, \delta_2 \rangle$ due automi a stati finiti, il loro prodotto $C = \langle \mathcal{S}, s_0, \mathcal{F}, \mathcal{L}, \delta \rangle$ è definito come segue:
    \begin{itemize}
        \item   $\mathcal{S} = \mathcal{S}_1 \times \mathcal{S}_2$
        \item $s_0 = (s_{01}, s_{02})$
        \item $\mathcal{F} = \{ (s_1, s_2) | s_1 \in \mathcal{F}_1 \land s_2 \in \mathcal{F}_2 \}$
        \item $\mathcal{L} = \mathcal{L}_1 \cap \mathcal{L}_2$
        \item $\delta$ è invece definita nel seguente modo:
    \end{itemize}
    \begin{equation*}
        \begin{cases}
            \text{$\delta ((s_1, s_2), a) = \{ (x,y) | x \in \delta(s_1, a) \land y \in \delta(s_2, a) \}$ \hfill se $\delta_1$ e $\delta_2$ sono definite} \\
            \text{non definito} \hfill \text{altrimenti}
        \end{cases}
    \end{equation*}
\end{definition}
\begin{remark}
    Nella definizione data sopra abbiamo definito l'automa prodotto che riconosce \l'\emph{intersezione} dei linguaggi riconosciuti dai due automi di partenza. Tuttavia è possibile anche definire l'automa prodotto che riconosce l'\emph{unione} dei due linguaggi invertendo opportunamente le varie congiunzioni (insiemistiche e logiche) con delle disgiunzioni.
\end{remark}
Per concludere mostriamo un esempio di prodotto tra due automi: i primi due automi in figura sono quelli di partenza e riconoscono rispettivamente le RegEx $ab(c^*)$ e $(ab)^*$ mentre il terzo è l'automa prodotto che riconosce la RegEx $ab$.
\begin{figure}[ht]
    \centering
    \begin{tikzpicture}[->,>=stealth',shorten >=1pt,auto,node distance=3cm,scale=1, transform shape, baseline=0, initial text=$ $,]
        % Automata 1
        \node[state, initial, initial text=$A$] (0) {$0$};
        \node[state] (1) [right of=0] {$1$};
        \node[state, accepting] (2) [right of=1] {$2$};
        % Automata 2
        \node[state, initial, , initial text=$B$, accepting] (3) [right of=2] {$0$};
        \node[state] (4) [right of=3] {$1$};
        % Product automata
        \node[state, initial, initial text=$A \times B$] (5) [below of=1, right of=0] {$0, 0$};
        \node[state] (6) [right of=5] {$1, 1$};
        \node[state, accepting] (7) [right of=6] {$2, 0$};

        \draw
        % Automata 1
        (0) edge node{$a$} (1)
        (1) edge node{$b$} (2)
        (2) edge[loop below] node{$c$} (2)
        % Automata 2
        (3) edge[bend left, above] node{$a$} (4)
        (4) edge[bend left, below] node{$b$} (3)
        % Product Automata
        (5) edge node{$a$} (6)
        (6) edge node{$b$} (7);
    \end{tikzpicture}
    \caption{Prodotto tra due automi}
\end{figure}

\section{Choreography Automata}
Passiamo ora alla definizione dei \emph{Choreography Automata} (CA); iniziamo diversificando la nozione di \emph{coreografia} e \emph{Choreography Automata}\cite{Choreography_Automata} il primo è un modello logico che permette di specificare le interazioni tra più attori (siano essi processi, programmi, etc.) all'interno di un sistema (concorrente nel nostro caso) mentre i secondi sono invece un'\emph{istanza} possibile per questo modello. In questo caso noi stiamo scegliendo di rappresentare le coreografie tramite degli Automi a Stati Finiti ma questo non esclude altre possibili realizzazioni.\\
Per prima cosa ricordiamo che le coreografie hanno due tipologie di \emph{view} possibili:
\begin{itemize}
    \item \textbf{Global View}: Che descrive il comportamento dei \emph{partecipanti} "as a whole" specificando anche come questi interagiscono tra loro.
    \item \textbf{Local View}: Che descrive il comportamento di un singolo partecipante in \emph{isolamento} rispetto agli altri.
\end{itemize}
La \emph{scelta implementativa} di utilizzare gli FSA è dovuta al fatto che gli stessi, oltre ad essere semplici ma espressivi, permettono di utilizzare loop "nested" ed "entangled" e permettono di sfruttare in maniera molto conveniente i risultati e le nozioni descritti in precedenza. I Choreography Automata sono dunque dei \emph{casi particolari} di automi a stati finiti in cui le transizioni specificano le interazioni tra i vari partecipanti della coreografia.\\
Un esempio di Choreography Automaton è visibile nella figura sottostante, la sintassi delle label sulle transizioni è la seguente: \emph{sender} $\rightarrow$ \emph{receiver} : \emph{message}.
\begin{figure}[ht]
    \centering
    \begin{tikzpicture}[->,>=stealth',shorten >=1pt,auto,node distance=3.8cm,scale=1, transform shape]
        \node[state,initial] (q0) {$q_0$};
        \node[state] (q1) [right of=q0] {$q_1$};
        \node[state] (q2) [right of=q1] {$q_2$};
        \draw
        (q0) edge[above] node{$A \rightarrow B: tic$} (q1)
        (q1) edge[above] node{$B \rightarrow C: count$} (q2)
        (q2) edge[bend left,above] node[below]{$C \rightarrow A: toc$} (q0);
    \end{tikzpicture}
    \caption{Un esempio di Choreography Automaton}
    \label{fig:ChoreographyAutomata_Example}
\end{figure}\\
In questo caso sono rappresentate le interazioni tra gli attori A,B e C, in particolare: A inizia la comunicazione mandando un messaggio \emph{tic} a B, B (dopo aver ricevuto tale messaggio) invia a sua volta \emph{count} a C ed infine C risponde ad A con messaggio \emph{toc}.

\begin{definition}[Choreography Automata]
    \label{def:Choreography_Automata}
    Un Choreography Automaton (c-automaton) è un $\epsilon$-free FSA con un insieme di label $\mathcal{L}_{int} = \{ A \rightarrow B : m | A \neq B \in \mathcal{P}, m \in \mathcal{M} \}$ dove:
    \begin{itemize}
        \item $\mathcal{P}$ è l'insieme dei partecipanti (per esempio A, B, ecc)
        \item $\mathcal{M}$ è l'insieme dei messaggi che possono essere scambiati (m, n, ecc)
    \end{itemize}
\end{definition}
\begin{remark}
    Anche se nella definizione non sono ammesse $\epsilon$-transizioni una variante non deterministica rimane sempre possibile.
\end{remark}

\subsection{CFSM e Local Views}
Ora che abbiamo una definizione formale dei Choreography Automata, possiamo concentrarci sull'estrapolazione delle varie view locali a partire dallo stesso. Ricordiamo che le view locali descrivono il comportamento di un singolo partecipante all'interno della coreografia e che sono ottenute attraverso un'operazione di \emph{proiezione} applicata all'intera coreografia (la view globale). Prima di definire però questa operazione di proiezione serve introdurre il concetto di \emph{Communicating Finite-State Machine (CFSM)}\cite{CFSM}. Come il nome suggerisce questo è sempre un modello basato su automi a stati finiti usato specificatamente per la descrizione delle local views. La principale differenza rispetto ai Choreography Automata sta nel fatto che le label sono \emph{direzionali}, ovvero possono essere del tipo "A B ? m" o "A B ! m" per indicare che A riceve (rispettivamente invia) un messaggio m a B.

\begin{definition}[Communicating Finite-State Machine] \label{def:CommunicatingFiniteStateMachine}
    Una Communicating Finite State Machine (CFSM) è un FSA $C$ con insieme di labels: \bigskip \\
    \centerline{$\mathcal{L}_{act} = \{$A B ! m, A B ? m $|$ A, B $ \in \mathcal{P},$ m $ \in \mathcal{M}\}$}
    \\ \\
    dove $\mathcal{P}$ e $\mathcal{M}$ sono definiti come in precedenza.
\end{definition}
Dunque il \emph{soggetto} di un'azione in input "A B ? m" è B, l'opposto vale per l'azione di output "A B ! m", indichiamo quindi con $M_a$ la CFSM che ha solo transizioni con soggetto A.\\
Ora che abbiamo introdotto tutti i concetti necessari possiamo definire di seguito l'operazione di \emph{Proiezione} su Choreography Automata.
Definiamo brevemente la notazione $s_1 \xrightarrow{a} s_{2}$ come abbreviazione per indicare che esiste una transizione da $s_1$ a $s_2$ con label $a$, formalmente $\exists a \in \mathcal{L}_{act}, s_1, s_2 \in \mathcal{S}. \delta(s_1, a) = s_2$.

\begin{definition}[Proiezione]
    La proiezione su A di una transizione $t = s_1$ $\xrightarrow{a}$ $s_2$ di un Choreography Automaton, scritta $t\downarrow_{A}$ è definita come:
    \begin{equation*}
        t \downarrow_{A} =
        \begin{cases}
            s \xrightarrow{\text{A C ! m}} s' & $se $ a = B \rightarrow C : m \wedge B = A       \\
            s \xrightarrow{\text{B A ? m}} s' & $se $ a = B \rightarrow C : m \wedge C = A       \\
            s \xrightarrow{\epsilon} s'       & $se $ a = B \rightarrow C : m \wedge B, C \neq A \\
            s \xrightarrow{\epsilon} s'       & $se $ a = \epsilon
        \end{cases}
    \end{equation*}
\end{definition}
\begin{remark}
    Si noti che esiste ed è possibile definire formalmente una funzione \emph{projection} che assegna ad ogni partecipante $p \in \mathcal{P}$ la sua CFSM $M_p$.
\end{remark}
La proiezione di un CA = $\langle \mathcal{S}, s_0, \mathcal{L}_{int}, \delta \rangle$ sul partecipante $p \in \mathcal{P}$, denotata con CA$\downarrow_p$ è ottenuta ricavando in primis l'automa intermedio: \bigskip \\
\centerline{$A_p = \langle \mathcal{S}, s_0, \mathcal{L}_{act}, \{ s \xrightarrow{t\downarrow_{p}} s' | s \xrightarrow{t} s' \in \delta \} \rangle$}
\\ \\
Tuttavia, come possiamo vedere nella definizione sopra, questo automa intermedio è non deterministico. È dunque necessario rimuovere le eventuali $\epsilon$ transizioni, ottenendo una versione deterministica e successivamente minimizzare quest'ultima. Entrambe le operazioni sono state definite rispettivamente negli algoritmi \ref{alg:SubsetConstruction_Algorithm} e \ref{alg:Minimization_Algorithm}

\begin{figure}[ht]
    \centering
    \begin{tikzpicture}[->,>=stealth',shorten >=1pt,auto,node distance=3.7cm,scale=1, transform shape]
        \node[state,initial,initial text=$CA\downarrow_A$] (A0) {0};
        \node[state] (A1) [below of=A0] {1,2};

        \node[state,initial,initial text=$CA\downarrow_B$] (B0) [right of=A0] {0,2};
        \node[state] (B1) [below of=B0] {1};

        \node[state,initial,initial text=$CA\downarrow_C$] (C0) [right of=B0] {0,1};
        \node[state] (C1) [below of=C0] {2};

        \draw
        (A0) edge[bend left, above] node[rotate=270]{A B ! tic} (A1)
        (A1) edge[bend left, above] node[rotate=90]{A C ? toc} (A0)

        (B0) edge[bend left, above] node[rotate=270]{B A ? tic} (B1)
        (B1) edge[bend left, above] node[rotate=90]{B C ! count} (B0)

        (C0) edge[bend left, above] node[rotate=270]{B C ? count} (C1)
        (C1) edge[bend left, above] node[rotate=90]{C A ! toc} (C0);
    \end{tikzpicture}
    \caption{Le tre view locali estratte dall'automa in figura \ref{fig:ChoreographyAutomata_Example}}
    \label{fig:Projection_Example}
\end{figure}

\subsection{Composizione delle global views}
Esiste anche un'operazione opposta alla proiezione, infatti è possibile \emph{comporre}\cite{CA_Composition} più Choreography Automata in uno unico che rappresenti le interazioni di tutti gli attori presenti. Questo può essere utile per vari motivi: potremmo avere, per esempio, delle \emph{global views locali} (composte da processi, thread o routine) che talvolta comunicano con altre \emph{global views remote} tramite delle \emph{interfacce} (come endpoint REST, WebSocket o connessioni TCP/IP). Da una situazione come questa può nascere l'esigenza di comporre insieme queste global views in una unica per visualizzare le interazioni (locali e non) che intercorrono tra i vari attori.\\
Introduciamo dunque l'operazione di \emph{composizione}, in questo caso assumiamo che gli insiemi dei partecipanti delle varie global views di partenza sia disgiunto in questo modo si evitano ambiguità nel risultato finale. La composizione si ottiene concatenando due operazioni:
\begin{itemize}
    \item \textbf{Prodotto} tra tutte le $n$ global views
    \item \textbf{Sincronizzazione} dell'automa prodotto precedentemente ottenuto
\end{itemize}
L'operazione di prodotto tra automi è stata enunciata nella definizione \ref{def:FSA_Product} e rimane pressochè invariata, l'operazione di \emph{sincronizzazione} è invece particolare: abbiamo appurato che le global views comunicano tra loro attraverso le interfacce, possiamo considerare quest'ultime come partecipanti alla coreografia con il solo ruolo di fare \emph{forwarding} dei messaggi tra una view e l'altra. Quindi ogni qualvolta la global view A vorrà mandare un messaggio a B, manderà un messaggio all'interfaccia I, lo stesso vale per B quando vorrà ricevere messaggi da A. La Sincronizzazione mira proprio a \emph{rimpiazzare} le interazioni che avvengono tramite interfacce con interazioni tra attori effettivi.\\\\
L'operazione di sincronizzazione genera un nuovo automa le cui label sono definite come segue:
\begin{equation*}
    \mathcal{S}(A \times B) =
    \begin{cases}
        p \xrightarrow{A \rightarrow B: m} r \hfill \text{ se } \exists p \xrightarrow{A \rightarrow H: m} q, \exists  q \xrightarrow{K \rightarrow B: m} r. (A \neq B) \\
        p \xrightarrow{A \rightarrow B: m} q \hfill \text{ se } A, B\in \mathcal{P}                                                                                     \\
        \text{nessuna transizione} \hfill \text{altrimenti}
    \end{cases}
\end{equation*}
Si noti che come step aggiuntivo alla trasformazione di sopra tutti gli stati non raggiungibili da quello iniziale verranno rimossi.\\
Vediamo di seguito l'esempio di una composizione tra due automi:
\begin{figure}[ht]
    \centering
    \begin{tikzpicture}[->,>=stealth',node distance=3.5cm,scale=1, transform shape, initial text=$ $,]
        \node[state, initial, initial text=$A$] (X0) {$0$};
        \node[state] (X1) [right of=X0] {$1$};
        \node[state] (X2) [right of=X1] {$2$};

        \node[state, initial, initial text=$B$] (Y0) [right of=X2] {$0$};
        \node[state] (Y1) [right of=Y0] {$1$};

        \node[state, initial, initial text=$\mathcal{S}(A \times B)$] (Z0) [below of=X1] {$0$};
        \node[state] (Z1) [right of=Z0] {$1$};
        \node[state] (Z2) [right of=Z1] {$2$};

        \draw
        (X0) edge node[above]{$A \rightarrow H : m$} (X1)
        (X1) edge node[above]{$I \rightarrow A : n$} (X2)
        (X2) edge[bend left, above] node{$A \rightarrow C : p$} (X0)

        (Y0) edge[bend left, above] node{$J \rightarrow B : m$} (Y1)
        (Y1) edge[bend left, above] node[below]{$B \rightarrow K : n$} (Y0)

        (Z0) edge node[above]{$A \rightarrow B : m$} (Z1)
        (Z1) edge node[above]{$B \rightarrow A : n$} (Z2)
        (Z2) edge[bend right, above] node{$A \rightarrow C : p$} (Z0);
    \end{tikzpicture}
    \caption{Un esempio di composizione tra due Choreography Automata}
    \label{fig:Composition_Example}
\end{figure}\\

\section{Analisi statica e dinamica}
Ora che abbiamo chiarito le nozioni di base per quanto riguarda la Teoria degli Automi e le coreografie, passiamo ad un'altro aspetto altrettanto importante per i fini di questa tesi. Considerando che l'obiettivo è quello di ottenere un Choreography Automaton partendo da un programma Go dobbiamo determinare in che modo è possibile estrarre delle informazioni da tale programma.\\
A questo riguardo ricordiamo che un programma può avere due \emph{formati}:
\begin{itemize}
    \item \textbf{Codice sorgente}: un testo scritto in un linguaggio \emph{human readable} con una specifica \emph{grammatica} e specifici \emph{costrutti} che descrivono, ad alto livello, i passi che devono essere intrapresi durante la computazione. Questo formato è quello più utilizzato dagli esseri umani in quanto più facile da comprendere (ed eventualmente modificare), tuttavia non è comprensibile ai calcolatori che, come sappiamo, lavorano con formati binari.
    \item \textbf{Codice binario}: detto anche codice macchina o codice eseguibile generato dal compilatore. Questo formato è difficilmente comprensibile da un umano, ma al contrario è perfettamente comprensibile per una macchina tant'è che può essere \emph{eseguito} dalla stessa.
\end{itemize}
Se consideriamo la definizione di programma come \emph{un insieme di istruzioni per arrivare ad un risultato finale partendo da input forniti} i formati suddetti sono due rappresentazioni equivalenti del medesimo programma e dunque possono essere usati intercambiabilmente e senza alterare la \emph{sostanza} del programma stesso. \bigskip \\
Tornando all'estrazione dei dati da un programma esistono due diverse di tecniche, legate al \emph{formato} del programma stesso:
\begin{itemize}
    \item \textbf{Analisi statica}\cite{Static_Analysis}: questo tipo di analisi viene eseguita sul codice sorgente, estraendo dei dati dallo stesso ma senza compilarlo nè eseguirlo. Questo tipo di analisi non considera e non è in grado di catturare il \emph{contesto d'esecuzione}, ovvero i fattori esterni che possono influenzare l'esecuzione di un programma a \emph{runtime}.
    \item \textbf{Analisi dinamica}\cite{Dynamic_Analysis}: questo tipo di analisi invece viene fatta attraverso la \emph{profilazione} del programma mentre lo stesso esegue, il programma è dunque in un formato binario. La profilazione può avvenire attraverso dei log emessi dal programma stesso oppure attraverso l'utilizzo di un'altro programma (detto \emph{tracer}) che controlla le operazioni eseguite dal programma target (il \emph{tracee})
\end{itemize}
Entrambe le tecniche presentano i rispettivi vantaggi e svantaggi: l'analisi statica permette una visione più completa in tempo più breve poichè osservando il codice sorgente riesce a catturare tutti i possibili percorsi in cui un programma potrebbe entrare, al contrario l'analisi dinamica non permette di avere sempre una visione completa in quanto è limitata ad osservare solo il percorso che l'esecuzione ha preso in quel momento.\\
Un esempio di questo comportamento può essere dato da un semplice costrutto come l'\texttt{if-then-else}: tramite l'analisi statica è possibile catturare con facilità entrambi i rami mentre tramite l'analisi dinamica è possibile solo osservare un ramo, quello che a runtime verifica la condizione specificata. Per questo specifico aspetto l'analisi dinamica restituisce dei dati \emph{parziali} e non è possibile fare assunzioni sul ramo che non è stato eseguito, tuttavia l'approssimazione di queste informazioni può essere migliorata eseguendo più profilazioni con input diversi. Si noti però che questo non è sufficiente a garantire che le informazioni siano complete ma solo meglio approssimate e il tempo richiesto per completare l'analisi diventa maggiore (proporzionale rispetto al numero di profilazioni).\\
In maniera opposta l'analisi statica non riesce a catturare completamente l'evoluzione del programma osservato nel tempo o l'influenza che fattori esterni quale il \emph{contesto d'esecuzione} abbiano sullo stesso, questi aspetti sono invece facilmente osservabili attraverso l'analisi dinamica. Anche in questo caso le informazioni restituite dall'analisi statica sono parziali e vanno approssimate, per esempio usando dei valori predefiniti.\bigskip \\
In conclusione, anche se le tecniche mostrate sopra prese singolarmente rappresentano un ottimo strumento per estrarre informazione da un programma, sono fondamentalmente complementari e andrebbero usate in combinazione per ottenere una visione \emph{completa} del programma stesso.

\subsection{Parsing e AST}
Appurato che l'analisi statica estrae i dati dal formato \emph{testuale} del programma, serve capire come è possibile ottenere informazioni dal codice sorgente Go. Il \emph{parsing} è l'operazione che permette di trasformare del codice sorgente in una struttura dati appropriata (l'\emph{Abstract Syntax Tree} o AST\cite{Abstract_Syntax_Tree}) dalla quale è poi possibile ricavare informazioni in maniera semplificata rispetto al dover utilizzare e manipolare la stringa iniziale (il contenuto testuale del file). Questa operazione non viene solo utilizzata nell'analisi statica ma è anche una fase importante del processo di compilazione (o interpretazione) di qualunque linguaggio di programmazione, il compilatore infatti può utilizzare l'AST per ottimizzare il codice sorgente e, in seguito, per generare il codice binario.\\
In generale, l'utilizzo di un AST fornisce vari vantaggi: il principale è quello di avere una struttura dati ben definita e gerarchica. Da questo consegue che è possibile navigare l'AST (in maniera molto simile ad una classica \emph{visita} su alberi) e questo permette di estrarre dati in maniera algoritmica dall'AST stesso. Inoltre, sempre grazie alla struttura gerarchica, è possibile definire delle trasformazioni per la stessa o verificare che rispetti certe proprietà.